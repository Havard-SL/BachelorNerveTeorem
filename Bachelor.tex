\documentclass[a4paper, titlepage, 12pt, norsk]{article}

\usepackage[utf8]{inputenc} % For å kunna skriva æøå i tekstar
\usepackage[T1]{fontenc}
\usepackage[margin=2cm]{geometry} % Fiksa margin
\usepackage[ddmmyyyy]{datetime} % Fiksar datorormatet på tiitelen
\usepackage{amssymb}
\usepackage{amsmath} % For visse mattesymbol, typ \mathbb
\usepackage{graphicx} % Bilete
%\usepackage{minted} % For kodesnuttar og resultat
\usepackage{enumitem} % Kan endra på korleis listar ser ut

\usepackage{amsthm}
\usepackage{thmtools} 

\usepackage{tikz}
\usetikzlibrary{matrix}

\RequirePackage{fontspec}
\setmainfont{Atkinson Hyperlegible}

% Ny type lista med ganske perfekt spacing
\newlist{plist}{enumerate}{5}
\setlist[plist]{align=left, itemindent = 0cm, labelsep = 0cm, labelindent = 0cm}
\setlist[plist,1]{label=\arabic*, font=\bf\Large}
\setlist[plist,2]{label*=.\arabic*, labelwidth=1.25cm, leftmargin=1.25cm}
\setlist[plist,3]{label*=.\arabic*, labelwidth=1.5cm, leftmargin=1.5cm}

\theoremstyle{plain}
\newtheorem{theorem}{Teorem}[section]
\newtheorem{proposition}[theorem]{Proposjon}
\newtheorem{corollary}[theorem]{Korollar}
\newtheorem{lemma}[theorem]{Lemma}

\theoremstyle{definition}
\newtheorem{definition}[theorem]{Definisjon}
\newtheorem{example}[theorem]{Eksempel}
\newtheorem{remark}[theorem]{Merknad}

\newcommand{\R}{\mathbb{R}}
\newcommand{\Q}{\mathbb{Q}}
\newcommand{\Z}{\mathbb{Z}}
\newcommand{\N}{\mathbb{N}}

\title{Nerveteorement og Anvendingar}
\author{Håvard Skjetne Lilleheie}

\begin{document}

\maketitle

\section{Grunnleggande definisjonar og resultat}

\begin{definition}
	For $V$ ei ikkje-tom og endeleg mengda, så definerar me $K$ til å vera eit \emph{endeleg asbtrakt simplisielt kompleks over $V$} om:
	\begin{enumerate}
		\item{$\forall v \in V: \{v\} \in K$}
		\item{$\forall \sigma \in K, \forall \tau \subseteq \sigma: \tau \in K$}
	\end{enumerate}
\end{definition}
Nokre eksempel av endelege abstrakte simplisielle kompleks over $\{p_1, p_2, p_3\}$:
\begin{itemize}
	\item{$A_1=\{\{p_1, p_2\}, \{p_1\}, \{p_2\}\}$}
	\item{$A_2=\{\{p_1\}, \{p_2\}, \{p_3\}, \{p_1, p_3\}, \{p_2, p_3\}$}
	\item{$A_3=\{\{p_1\}, \{p_2\}, \{p_3\}\}$}
\end{itemize}
Og her er nokre ikkje-eksempel av endelege abstrakte simplisielle kompleks over $\{p_1, p_2, p_3\}$:
\begin{itemize}
	\item{$A_1'=\{\{p_1, p_2\}, \{p_1\}\}=A_1 \setminus \{\{p_2\}\}$}
	\item{$A_2'=\{\{p_1\}, \{p_2\}, \{p_3\}, \{p_1, p_3\}, \{p_2, p_3\}, \{p_1, p_2, p_3\}\}=A_2 \cup \{\{p_1, p_2, p_3\}\}$}
	\item{$A_3'=\{\{p_1\}, \{p_2\}, \{p_2, p_3\}\}=\left(A_3 \setminus \{\{p_3\}\}\right) \cup \{\{p_2, p_3\}\}$}
\end{itemize}
Inspirasjonen bak denne definisjonen kan verka tilfeldig og heilt umotivert, men som me kjem til å sjå seinare; så er dette ein svært praktisk simplifisering av informasjon som bevarer mykje av eigenskapane til den originale datamengda ein vil sjå på. 
\\Fyrst, lat oss sjå korleis dette kan tenkast på som meir geometrisk:
\begin{definition}
	Ein endeleg mengda punkt $\{p_1, p_2, p_3, \dots, p_n\}$ i $\R^m$ for $m\geq1$ er \emph{geometrisk uavhengige} om for $\{t_i\}_{i=0}^n$ med $t_i\in\R$, så:
	\begin{equation*}
		\left(\sum_{i=0}^n t_i=0 \land  \sum_{i_0}^n t_ip_i=0\right)\Rightarrow t_i=0 \; \forall i
	\end{equation*}
\end{definition}
\begin{theorem}
	Ei endeleg mengda $\{p_1, p_2, p_3, \dots, p_n \}$ av punkt i $\R^m$ er gemetrisk uavhengige $\Leftrightarrow$ vektorane $\{(p_2-p_1), (p_3-p_1), (p_4-p_1),\dots,(p_n-p_1)\}$ er lineært uavhengige i $\R^m$
\end{theorem}
\begin{proof}
	(\Rightarrow)
	\\For $a_i\in\R$, annta $\sum_{i=2}^na_i(p_i-p_1)=0$. Det kan me skriva om til: $\sum_{i=2}^na_ip_i + \left(-\sum_{i=2}^na_i\right)p_1$. Om me då definerar: $a_1 := -\sum_{i=2}^na_i$, så ser me at $\sum_{i=1}^na_i=\sum_{i=2}^na_i-\sum_{i=2}^na_i=0$, og at $\sum_{i=1}^na_ip_i=\sum_{i=2}^na_i(p_i-p_1)=0$. Og sidan $\{p_i\}_{i=1}^n$ er geometrisk uavhengig frå antaginga, så $\Rightarrow$ $a_i=0 \, \forall i$ 	
	\\(\Leftarrow)
	\\Mutt
\end{proof}
\begin{definition}
	Konveks kombinasjon
\end{definition}
\begin{definition}
	Geometrisk simplex
\end{definition}
\begin{definition}
	Simplisielt kompleks
\end{definition}
\begin{definition}
	Affin imbedding
\end{definition}
\begin{definition}
	Ein \emph{geometriske realisering} til eit endeleg abstrakt simplisielt kompleks over ei ikkje-tom mengda $V$ er, gitt ein vilkårleg affin imbedding $f:V\to\R^n$ for ein eller annan $n\geq1$, 
\end{definition}
\begin{theorem}
	geometrisk realisering unik opp til homeomorfi
\end{theorem}
\begin{remark}
	Derfor me bruker bestemt form
\end{remark}



\end{document}
