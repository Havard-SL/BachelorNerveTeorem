\documentclass[a4paper, titlepage, 12pt, norsk]{article}

% For å kunna skriva æøå i tekstar MERK: Blir automatisk ubrukeleg med lualatex of fontspec
%\usepackage[utf8]{inputenc} 
\usepackage[T1]{fontenc}
\usepackage[margin=2cm]{geometry} % Fiksa margin
\usepackage[ddmmyyyy]{datetime} % Fiksar datorormatet på tiitelen
\usepackage{amssymb}
\usepackage{amsmath} % For visse mattesymbol, typ \mathbb
\usepackage{graphicx} % Bilete
%\usepackage{minted} % For kodesnuttar og resultat
\usepackage{enumitem} % Kan endra på korleis listar ser ut

\usepackage[hidelinks,colorlinks=true]{hyperref} % For autoref
\hypersetup{allcolors=[rgb]{0,0.31,0.62}} % For fargar på ting ein referer til i autoref

\usepackage{amsthm} % For teorem, definisjon, bevis enviornments.
\usepackage{thmtools} 

\usepackage{svg}
\svgpath{svg/}
\usepackage{float}
%\usepackage[subsection]{placeins}

%\usepackage{tikz}
%\usetikzlibrary{matrix}
%\newcommand{\diagram}[3]{\matrix (#1) [matrix of math nodes,row
%  sep={#2},column sep={#3},text height=1.5ex,text
%  depth=0.25ex]}

\usepackage{tikz-cd} %For kommutative diagram med tikz
\usepackage{parskip} %Fjernar indents, men gjer linjeavstanden kortare (Kanskje)

\usepackage{fontspec}
\usepackage{unicode-math}

%\setmainfont{Atkinson Hyperlegible}
\setmainfont{Fira Sans}
\setmathfont{Fira Math}
\setmathfont[range=\setminus]{Asana Math}

% Ny type lista med ganske perfekt spacing
\newlist{plist}{enumerate}{5}
\setlist[plist]{align=left, itemindent = 0cm, labelsep = 0cm, labelindent = 0cm}
\setlist[plist,1]{label=\arabic*, font=\bf\Large}
\setlist[plist,2]{label*=.\arabic*, labelwidth=1.25cm, leftmargin=1.25cm}
\setlist[plist,3]{label*=.\arabic*, labelwidth=1.5cm, leftmargin=1.5cm}

\theoremstyle{plain}
\newtheorem{theorem}{Teorem}[section]
\newtheorem{proposition}[theorem]{Proposjon}
\newtheorem{corollary}[theorem]{Korollar}
\newtheorem{lemma}[theorem]{Lemma}

\theoremstyle{definition}
\newtheorem{definition}[theorem]{Definisjon}
\newtheorem{example}[theorem]{Døme}
\newtheorem{remark}[theorem]{Merknad}

\usepackage[norsk]{babel}
%\renewcommand*{\proofname}{Bevis}

\newcommand{\Rb}{\mathbb{R}}
\newcommand{\Qb}{\mathbb{Q}}
\newcommand{\Zb}{\mathbb{Z}}
\newcommand{\Nb}{\mathbb{N}}
\newcommand{\Nc}{\mathcal{N}}
%\newcommand{\Cech}{ \operatorname{Cech} }
\DeclareMathOperator{\Cech}{Cech}
\newcommand{\intersect}{ \mathop{\cap}\limits } % For finare grenser på snitt
\newcommand{\union}{ \mathop{\cup}\limits }
\newcommand{\gr}[1]{ \lvert #1 \rvert } % Geometrisk realisering
\newcommand{\set}[1]{ \left \{ #1 \right \} } % Mengda
\newcommand{\tuple}[1]{ \left( #1 \right) } % Tupel
\DeclareMathOperator{\Sd}{Sd}
\DeclareMathOperator{\bst}{bst}
\DeclareMathOperator{\Sk}{Sk}

% PERSONLEGE NOTAT

% Større separator i mengdar
% Rydda vekk semikolonar og kolonar i teksten min
% \cup og \cap ser merkelege ut
% Bytta font? Litt uleseleg til tider. Ingen kaligrafi N, \bigcup/\bigcap f.eks
% Fiksa tikz diagrammet
% Skriva om det longe beviset
% Fleire eksempel!!
% Kva er feil med \vert og |?
% Nervedefinisjon for topologisk rom og kun endeleg?
% Hugs å spesifisér alt! Skriv ned idéen bak korleis eg skriv ting.
% Seksjon 0 med definisjonar og minimumskrav?
% Figurtekst skal vera kort, eksempelteksten skal referera til figurteksten
% Eksempeltekst til kvar figur / figur må bli nemd minst ein gong.
% Skriv om beviset for 1.17 (homeomorfi mellom geometriske realiseringar.)
% Legg til intuitiv forklaring på kva som foregår i byrjinga.
% Kun euklidsk norm på geometrisk realisering.

% PLAN
% Fokus på å forbetra del 1, og så byrja på del 2. Pass på Magnus sitt bevis er litt shady.

% MERK:
% Drit i uendeleg!


\title{Nerveteoremet og Anvendingar}
\author{Håvard Skjetne Lilleheie}

\begin{document}

\maketitle

\section{Førehandskunnskapar og teorem}

\subsection{Notasjon}

La \( \#S \) vera antal element i \( S \)

La \( \subset \) vera streng inklusjon, og la \( \subseteq \) vera inklusjon eller lik.

\subsection{Førehandskunnskapar}

Grunnleggande topologi

Grunnleggande lineær algebra

\subsection{Viktige teorem}

\begin{theorem} \label{thm:begrensa-lin-op-er-kont}
	Begrensa lin operatorar er kont.
\end{theorem}

\begin{theorem} \label{thm:definer-lin-op}
	Kan definera lin operator på n lineært uavhengige vektorar
\end{theorem}

\begin{theorem} \label{thm:universal-eigenskap-underromstopologi}
	Universaleigenskapen til underromstopologien (Brukt for å visa at lineære restrikterte avbildningar er kont.)
\end{theorem}

\begin{theorem} \label{thm:closed-map-lemma}
	Closed map lemma (Visa at avb er lukka)
\end{theorem}

\begin{theorem} \label{thm:heine-borel}
	Heine borel (Visa at Geometriske simpleks er kompakte)
\end{theorem}

\begin{theorem}
	Pasting lemma
\end{theorem}

\begin{theorem} \label{thm:distanse-er-kont}
	d er kont
\end{theorem}

\begin{theorem} \label{thm:maksimum-av-kont-er-kont}
	maksimum av kont funksjonar er kont
\end{theorem}

\begin{theorem} \label{thm:infimum-over-kompakt-er-min}
	infimum over kompakt er min
\end{theorem}

\begin{theorem} \label{thm:supremum-over-kompakt-er-maks}
	supremum over kompakt er max
\end{theorem}


\begin{theorem} \label{thm:endeleg-union-kompakt-er-kompakt}
	Endeleg union kompakt mengde er kompakt
\end{theorem}

\section{Grunnleggande definisjonar og resultat}

\begin{definition} \label{def:ASK}
	For $V$ ei ikkje-tom og endeleg mengda, så definerar me $K$ til å vera eit \emph{endeleg asbtrakt simplisielt kompleks over $V$} om:
	\begin{enumerate}
		\item{$\forall v \in V: \{v\} \in K$}
		\item{\( \forall \sigma \in K: \sigma \subseteq V \)}
		\item{$\forall \sigma \in K, \forall \tau \subseteq \sigma: \tau \in K$}
	\end{enumerate}
	Då kallar me og $V$ for \emph{hjørnemengda} til $K$.
\end{definition}

Nokre eksempel av endelege abstrakte simplisielle kompleks over \( P=\set{p_1, p_2, p_3} \):

\begin{itemize}
	\item{$A_1=\{\{p_1, p_2\}, \{p_1\}, \{p_2\}\}$}
	\item{$A_2=\{\{p_1\}, \{p_2\}, \{p_3\}, \{p_1, p_3\}, \{p_2, p_3\}\}$}
	\item{$A_3=\{\{p_1\}, \{p_2\}, \{p_3\}\}$}
\end{itemize}

Og her er nokre ikkje-eksempel av endelege abstrakte simplisielle kompleks over \( P=\set{p_1, p_2, p_3} \):

\begin{itemize}
	\item{$A_1'=\{\{p_1, p_2\}, \{p_1\}\}=A_1 \setminus \{\{p_2\}\}$}
	\item{$A_2'=\{\{p_1\}, \{p_2\}, \{p_3\}, \{p_1, p_3\}, \{p_2, p_3\}, \{p_1, p_2, p_3\}\}=A_2 \union \{\{p_1, p_2, p_3\}\}$}
	\item{$A_3'=\{\{p_1\}, \{p_2\}, \{p_2, p_3\}\}=\left(A_3 \setminus \{\{p_3\}\}\right) \union \{\{p_2, p_3\}\}$}
\end{itemize}

Inspirasjonen bak denne definisjonen kan verka tilfeldig og heilt umotivert, men som me kjem til å sjå seinare; så er dette ein svært praktisk simplifisering av informasjon som bevarer mykje av eigenskapane til den originale datamengda ein vil sjå på.

Fyrst, lat oss sjå korleis dette kan tenkast på som meir geometrisk:

\begin{definition}
	Ein endeleg mengda punkt $\{p_1, p_2, p_3, \dots, p_n\}$ i $\Rb^m$ for $m\geq1$ er \emph{geometrisk uavhengige} om for $\{a_i\}_{i=0}^n$ med $a_i\in\Rb$, så:
	\begin{equation*}
		\left(\sum_{i=1}^n a_i=0 \land  \sum_{i=1}^n a_ip_i=0\right)\implies a_i=0 \; \forall i
	\end{equation*}
\end{definition}

\begin{theorem}\label{thm:geometrisklineærtuavhengig}
	Ei endeleg mengda $\{p_1, p_2, p_3, \dots, p_n \}$ av punkt i $\Rb^m$ er geometrisk uavhengige $\Longleftrightarrow$ vektorane $\{(p_2-p_1), (p_3-p_1), (p_4-p_1),\dots,(p_n-p_1)\}$ er lineært uavhengige i $\Rb^m$
\end{theorem}

\begin{proof}
	($\implies$)
	\\For $a_i\in\Rb$, annta $\sum_{i=2}^na_i(p_i-p_1)=0$. Om me då definerar: $a_1 := -\sum_{i=2}^na_i$, så ser me at 
	\begin{equation*}
		\sum_{i=1}^na_i=\sum_{i=2}^na_i-\sum_{i=2}^na_i=0
	\end{equation*}
	og at 
	\begin{equation*}
		\sum_{i=1}^na_ip_i=\sum_{i=2}^na_i(p_i-p_1)=0
	\end{equation*}
	Og sidan $\{p_i\}_{i=1}^n$ er geometrisk uavhengig frå antaginga, så $\implies$ $a_i=0 \, \forall i$. Som er definisjonen på lineært uavhengig.
	\\(\(\Longleftarrow\))
	\\for $a_i\in\Rb$, annta $\sum_{i=1}^n a_i=0$ og $\sum_{i=1}^n a_ip_i=0$. Då ser me at 
	\begin{equation*}
		a_1=-\sum_{i=2}^n a_i
	\end{equation*} 
	Det gir oss: 
	\begin{equation*}
		0=\sum_{i=1}^n a_ip_i=\sum_{i=2}^n a_ip_i-\sum_{i=2}a_ip_1=\sum_{i=2}a_i(p_i-p_1)
	\end{equation*}
	Men sidan $\{(p_i-p_1)\}_{i=2}^n$ er lineært uavhengig, så: $\implies a_i = 0$ for $i\in[2,n]$. Men sidan $a_1 = -\sum_{i=2}^n a_i=0$, så får me $a_i=0 \forall i$. Som er definisjonen på geometrisk uavhengig.
\end{proof}

\begin{remark}
	Ein direkte konsekvens frå dette resultatet og grunnleggande lineær algebra er at ein geometrisk uavhengig mengda i $\Rb^m$ kan maksimalt innehalda $m+1$ forskjellige punkt. Dette er fordi det kan ikkje vera meir enn $m$ lineært uavhengige vektorar i eit $m$-dimensjonalt vektorrom.
\end{remark}

\begin{definition}
	Ein \emph{konveks kombinasjon} av ei endeleg mengda \( P = \set{p_1, p_2, p_3, \dots, p_n} \) av punkt i \( \Rb^m \) er eit punkt \( x\in\Rb^m \) gitt ein tupel av koeffisientar \( A=\{a_1, a_2, a_3, \dots, a_n\} \) i \( \Rb \), med \( a_i\geq0\forall i \) og \( \sum_{i=1}^n a_i = 1 : \)
	
	\begin{equation*}
		x=\sum_{i=1}^n a_ip_i
	\end{equation*}
	Om $P$ er ein tupel (ein ordna mengda) så blir koeffisientane også ein tupel, og $A$ blir då kalla dei \emph{barysentriske koordinatane} til $x$.
\end{definition}

\begin{theorem} \label{thm:unik-barysentrisk-koordinat}
	Dei barysentriske koordinatane til ein konveks kombinasjon av geometrisk uavhengige punkt er eintydig.
\end{theorem}

\begin{proof}
	La $x$ vera ein konveks kombinasjon av geometrisk uavhengige punkt \( \tuple{p_1, p_2, \dots, p_n} \) i $\Rb^m$. Annta at $x$ har to barysentriske koordinatar: $(a_1, a_2, a_3, \dots, a_n)$ og $(b_1, b_2, b_3, \dots, b_n)$. Det betyr at:
	\begin{equation*}
		0 = x - x = \sum_{i=1}^n a_ip_i - \sum_{i=1}^n b_ip_i=\sum_{i=1}^n (a_i-b_i)p_i
	\end{equation*}
	og me får:
	\begin{equation*}
		\sum_{i=1}^n(a_i-b_i)=\sum_{i=1}^na_i - \sum_{i=1}^nb_i = 1 - 1 = 0
	\end{equation*}
	Men sidan $(p_1, p_2, p_3, \dots, p_n)$ er geometrisk uavhengig, så 
	\begin{equation*}
		\implies (a_i-b_i)=0\forall i \Longleftrightarrow a_i = b_i \forall i
	\end{equation*}
	og dei barysentriske koordinatane er derfor like.
\end{proof}

\begin{definition}
	Det \emph{geometrisk simplekset} utspent av ei geometrisk uavhengig mengda av punkt $P=\{p_1, p_2, p_3, \dots, p_n\}$ i $\Rb^m$, er alle konvekse kombinasjonar av $P$.
	\\Vidare, så kallar me dette geometriske simplekset for eit \emph{$(n-1)$-simpleks}, når $P$ består av $n$ punkt.
\end{definition}

\begin{remark}
	Ut ifrå den førre definisjonen så ser me at eit $0$-simpleks kun er eit enkelt punkt, eit $1$-simpleks er alle konvekse kombinasjonar mellom to punkt, som viser seg å vera ei linjestykket mellom dei to punkta. Og eit $2$-simpleks dannar ein trekant. Ein $3$-simpleks blir også kalla eit tetraeder. Denne simpleks definisjonenen gir ein fin matematisk forklaring av "trekant"-strukturar i $\Rb^m$.
\end{remark}

\begin{example}
	\phantom{123}
	\begin{figure}[htbp]
		\begin{center}
			\includesvg[width=0.6\textwidth]{Eksempel1-8.svg}
			\caption{Her er tre geometriske simpleksar. Ein \( 1 \)-simpleks utspunne av \( \set{a_1, a_2} \), Ein \(2\)-simpleks utspunne av \( \set{b_1, b_2, b_3} \), og ein \(3\)-simpleks utspunne av \( \set{c_1, c_2, c_3, c_4} \)}
		\end{center}
	\end{figure}
\end{example}

\begin{remark}
	Noko som er svært interessant med denne definisjonen er at om ein har ei geometrisk uavhengig mengda $P$, og ser på $\hat{P}_i := P \setminus \{p_i\}$, så vil dette også vera ei geometrisk uavhengig mengda. I tilleg så vil det geometriske simplekset utspent av $\hat{P}_i$ vere ei delmengda av det geometriske simplekset utspent av $P$. Med andre ord; alle delmengdar av $P$ dannar også andre geometriske simpleks, inneholdt i det originale geometriske simplekset! Men ikkje nok med det, fordi når ein ser på det geometriske simplekset utspent av $\hat{P}_i$ for ein eller annan vilkårleg $i$, så ser me at det er som eine "fjeset" av det geometriske simplekset utspunne av $P$. 
	%Dette inspirerar ein ny definisjon:
\end{remark}

% \begin{definition}
% 	Eit \emph{fjes} til eit geometrisk simpleks utspunne av ein geometrisk uavhengig mengda $P=\{p_1, p_2, p_3, \dots, p_n\}$, er eit geometriske simpleks utspunne av $\hat{P}_i := P\setminus \{p_i\}$ for ein eller annan $i$.
% \end{definition}

% \begin{definition} % Ikkje utspunne av nokre punkt?
% 	Eit \emph{geometrisk simplisielt kompleks} er ei mengda av geometriske simpleksar (utspent av punkt i $\Rb^m$), $K$, sånn at:
% 	\begin{enumerate}
% 		\item{For $\sigma \in K$ så er alle fjesa til $\sigma$ også i $K$.}
% 		\item{For $\sigma, \tau \in K$ med $\sigma \intersect \tau \neq \emptyset$, då er $\sigma \intersect \tau$ eit fjes av både $\sigma$ og $\tau$.}
% 	\end{enumerate}
% \end{definition}

% \begin{theorem} \label{thm:unik-utvida-barysentrisk-koordinat}
% 	TODO Geometrisk simplisielle kompleks har ein utvida unik barysentrisk koordinat.
% \end{theorem}

\begin{definition}
	For $V$ ein endeleg mengda og $m\geq1$, så er $f:V\rightarrow \Rb^m$ ein \emph{affin imbedding} om $f$ er injektiv, og om biletet, $f(V)$ er ei geometrisk uavhengig mengda av punkt.
\end{definition}

\begin{definition}
	For eit endeleg abstrakt simplisielt kompleks $K$, over ei ikkje-tom hjørnemengda $V$, gitt ein vilkårleg affin imbedding $f:V\to\Rb^m$ for ein eller annan $m\geq1$. Så er den \emph{geometriske realiseringa med hensyn til $f$ av \( K \)} unionen av alle dei geometriske simpleksane utspunne av punkta i $f(\sigma)$, for $\sigma\in K$. Dette er ofte betegna $\gr{K}_f$
\end{definition}

% \begin{theorem}
% 	TODO geometrisk realisering er eit geometrisk simplisielt kompleks
% \end{theorem}

\begin{lemma} \label{thm:tau-homeomorfi}
	Gitt \( K \) eit abstrakt simplisielt kompleks over \( V \), og \( f \) ein affin imbedding \( f: V \to \Rb^n \). Fiksér \( x \in \Rb^n \). For eit vilkårleg punkt \( y \in \Rb^n \), definer avbildninga \( \hat{\tau}: y \mapsto y-x \). Då er \( \hat{\tau} \circ f \) ein affin imbedding, og \( \hat{\tau} \) ein homeomorfi frå \( \gr{K}_f \) til \( \gr{K}_{\hat{\tau} \circ f} \).
\end{lemma}

\begin{proof} %Burde visa at bevarar barysentriske koordinatar også?
	For å visa at \( \hat{\tau} \circ f \) er ein affin imbedding, så ser me på \( \hat{\tau} \circ f(V) = (f(v_1)-x, f(v_2)-x, \dots, f(v_n)-x) \). Me får då at \( (f(v_2)-x-(f(v_1)-x), f(v_3)-x-(f(v_1)-x), \dots, f(v_n)-x-(f(v_1)-x)) = ( f(v_2)-f(v_1), f(v_3)-f(v_1), \dots, f(v_n)-f(v_1) ) \). Og frå \autoref{thm:geometrisklineærtuavhengig}, så får me at \( ( f(v_2)-f(v_1), f(v_3)-f(v_1), \dots, f(v_n)-f(v_1) ) \) er lineært uavhengig fordi \( f(V) \) er geometrisk uavhengig, men det betyr igjen at \( (f(v_2)-x-(f(v_1)-x), f(v_3)-x-(f(v_1)-x), \dots, f(v_n)-x-(f(v_1)-x)) \) er lineært uavhengig, som betyr at \( \hat{\tau} \circ f(V) \) er geometrisk uavhengig.

	For å visa at \( \hat{\tau}(\gr{K}_f) \subseteq \gr{K}_{\hat{\tau} \circ f} \), så ser me på \( y \in \hat{\tau}(\gr{K}_f) \). Då er \( y \) ein konveks kombinasjon av \( f(\sigma) \) for ein eller annan \( \sigma \in K \). Det vil seie at for \( k \) lik antal element i \( f(\sigma) \), så er \( y = \sum_{i=1}^k a_i f(v_i) \). Då er
	\begin{align*}
		\hat{\tau(y)} &= \left(\sum_{i=1}^k a_i f(v_i)\right) - x \\
		&= \left(\sum_{i=1}^k a_i f(v_i)\right)-\left(\sum_{i=1}^k a_i \right)x \\
		&= \left(\sum_{i=1}^k a_i f(v_i)\right)-\sum_{i=1}^k a_i x \\
		&= \sum_{i=1}^k a_i f(v_i) - a_i x \\
		&= \sum_{i=1}^k a_i(f(v_i) - x)
	\end{align*}
	Som er ein konveks kombinasjon av element i \( \hat{\tau} \circ f(\sigma) \), og derfor ei delmengda av \( \gr{K}_{\hat{\tau} \circ f} \).

	For å visa at \( \hat{\tau}(\gr{K}_f) \supseteq \gr{K}_{\hat{\tau} \circ f} \), så tek me \( y \in \gr{K}_{\hat{\tau} \circ f} \). Då, for ein eller annan \( \sigma \in K \), så er \( y \) ein konveks kombinasjon av element i \( \hat{\tau}\circ f(\sigma) \). Altså: \(  y = \sum_{i=1}^k a_i (f(v_i) - x \). Ein lik sum-manipulasjon som over motsatt veg gir oss at \( y = \hat{\tau}\left(\sum_{i=1}^k a_i f(v_i)\right) \), altså \( \hat{\tau}(z) \), der \( z \) er ein konveks kombinasjon av punkta \( f(\sigma) \). Altså eit element i \( \hat{\tau}(\gr{K}_f) \).

	\( \hat{\tau} \) er kontinuerlig fordi \( \hat{\tau} = \tau \circ i \) for \( \tau: \Rb^n \to \Rb^n \) som tek \( y \mapsto y-x \) er ein homeomorfi, og \( i: \gr{K}_f \hookrightarrow Rb^n \), inklusjonen inn i \( \Rb^n \) er kontinuerlig fordi \( \gr{K}_f \subseteq \Rb^n \) har underromstopologien. Så \( \hat{\tau} \) er ein komposisjon av to kontinuerlige funksjonar og derfor kontinuerlig sjølve.

	Av symmetri så kan eit likt argument kan bli gjort på \( \hat{\tau}^{-1}: y \mapsto y+x \) og ein får derfor at \( \hat{\tau} \) er ein homeomorfi.
\end{proof}

\begin{theorem}
	Geometrisk realisering er eintydig opp til homeomorfi. Med andre ord: For to ulike geometriske realiseringar av eit endeleg abstrakt simplisielt kompleks $K$, med hensyn til $f$ og $g$ (to ulike affine imbeddingar), så er $\gr{K}_f$ og $\gr{K}_g$ homeomorfe.
\end{theorem}

\begin{proof}%Rotete? Trenge unike barysentriske koordinatar? Må visa \hat{\tau} er bijeksjonar
	La $K$ vera eit abstrakt simplisielt kompleks, og la $f:K\to\Rb^m$ og $g:K\to\Rb^l$ vera to affine imbeddingar. La $V=\{ v_1, v_2, \dots, v_n \}$ vera hjørnemengda til $K$ Definér vidare $x_i=(f(k_{i-1})-f(k_1))$ og $y_i=(g(k_{i-1})-g(k_1))$. La $\tau_f:\Rb^m\to\Rb^m$ vera ein forskyving som tek $x\mapsto x-f(k_1)$. Og la $\tau_g:\Rb^l\to\Rb^l$ vera ein forskyving som tek $x\mapsto x-g(k_1)$. Til slutt, la $L_1:\Rb^m\to\Rb^l$ vera den lineære funksjonen som sender $x_i\mapsto y_i$.
	Og for praktiske grunnar: Definér $\hat{f}:=\tau_f\circ f$ og $\hat{g}:=\tau_g \circ g$
Me får då at $L_1(\hat{f}(k_i))=\hat{g}(k_i)$, fordi:
For $i=1$:

	\begin{equation*}
		L_1(\hat{f}(k_1)=L_1(f(k_1)-f(k_1))=L(0)=0=g(k_1)-g(k_1)=\hat{g}(k_1)
	\end{equation*}

	For $i\neq 1$:

	\begin{equation*}
		L_1(\hat{f}(k_i))=L_1(f(k_i)-f(k_1))=L_1(x_{i-1})=y_{i-1}=g(k_i)-g(k_1)=\hat{g}(k_i)
	\end{equation*}
	Om ein då let $\hat{L}=L|_{\gr{K}_{\hat{f}}}$, og $L_2:\Rb^l\to\Rb^m$ vera den lineære funskjonen som sender $y_i\mapsto x_i$ og let $\tilde{L}=L_2|_{\gr{K}_{\hat{g}}}$, så er definisjonane symmetriske og det symmetriske resultatet gjelder derfor for $L_2$ også.

	For eit vilkårleg element $x\in\gr{K}_{\hat{f}}$ så er $x$ eit element av ein simpleks i det geometriske simplisielle komplekset. Det vil seie, me kan utrykka $x$ som ein konveks sum av ein geometrisk uavhengig mengda, som korresponderar til den simpleksen $x$ er eit element i.
	
	Me velger derfor ein $\sigma\in K$ sånn at $\hat{f}(\sigma)=\{\hat{f}(k_1), \hat{f}(k_2), \dots, \hat{f}(k_r)\}$. Dette er ein geometrisk uavhengig mengda fordi $\tau_f$ er injektiv og sender $x_i\mapsto x_i$ som er lineært uavhengig og biletet er derfor geometrisk uavhengig om $\{x_i\}$ er lineært uavhengig, som det er frå \autoref{thm:geometrisklineærtuavhengig}.
	
	La denne mengda utspenna punkta til ein simpleks som $x$ er eit element av. Då kan me utrykka $x$ som ein konveks kombinasjon av elementa i $\hat{f}(\sigma)$: $x=\sum_{i=1}^ka_i\hat{f}(k_i)$ med $\sum_{i=1}^ka_i=1$ og $a_i\geq0\; \forall i$. 
	
	Her er valet av kva geometrisk simpleks me velger ikkje vikitg, ettersom $\hat{L}$ er veldefinert.
	\begin{align*}
		\tilde{L}\circ \hat{L}(x) &= \tilde{L}\circ \hat{L}\left(\sum_{i=1}^ra_i\hat{f}(k_i)\right) \\
		&= \tilde{L}\left(\sum_{i=1}^r\hat{L}(a_i\hat{f}(k_i))\right) \\
		&= \tilde{L}\left(\sum_{i=1}^ra_i\hat{L}(\hat{f}(k_i))\right) \\
		&= \tilde{L}\left(\sum_{i=1}^ra_i\hat{g}(k_i))\right) \\
		\intertext{Sidan $\sum_{i=1}^ra_i\hat{g}(k_i)\in\gr{K}_{\hat{g}}$, så får me:} \\
		&= \sum_{i=1}^r\tilde{L}(a_i\hat{g}(k_i)) \\
		&= \sum_{i=1}^ra_i\tilde{L}(\hat{g}(k_i)) \\
		&= \sum_{i=1}^ra_i\hat{f}(k_i) \\
		&= x
	\end{align*}
	Og likt for $\hat{L}\circ\tilde{L}(y)$.
	Så det betyr at $\hat{L}$ er bijektiv med invers $\tilde{L}$.
	
	Om ein let $\hat{\tau}_f:=\tau_f|_{\gr{K}_f}$ og $\hat{\tau}_g:=\tau_g|_{\gr{K}_g}$, så kan ein sjå at $\hat{\tau}_f\gr{K}_f\to\gr{K}_{\hat{f}}$ er ein bijeksjon, og likt for $\hat{\tau}_g$.
	
	Då får ein følgande kommutative diagram:
%	\begin{center}
%		\begin{tikzpicture}
%			\diagram{d}{2.5em}{2.5em}{
%				\Rb^m & \Rb^m & \Rb^l & \Rb^l \\
%				\vert K\vert_f & \vert K\vert_{\hat{f}} & \vert K\vert_{\hat{g}} & \vert K\vert_g \\
%				};
%		\path[->,font = \scriptsize, midway]
%		(d-1-1) edge node[above]{$\tau_f$} (d-1-2)
%		(d-1-2) edge node[above]{$L$} (d-1-3)
%		(d-1-4) edge node[above]{$\tau_g$} (d-1-3)
%		(d-2-1) edge [dashed,->] node[below]{$\hat{\tau}_f$}(d-2-2)
%		(d-2-2) edge node[below]{$\hat{L}$} (d-2-3)
%		(d-2-4) edge [dashed,->] node[below]{$\hat{\tau}_g$} (d-2-3)
%		(d-2-1) edge (d-1-1)
%		(d-2-2) edge (d-1-2)
%		(d-2-3) edge (d-1-3)
%		(d-2-4) edge (d-1-4);
%		\end{tikzpicture}
%	\end{center}
	\begin{center} % |, \vert funke ikkje??
		\begin{tikzcd}
			\Rb^m \arrow{r}{\tau_f}
			& \Rb^m \arrow{r}{L}
			& \Rb^l
			& \Rb^l \arrow{l}[swap]{\tau_g} \\
			\gr{K}_f \arrow[dashed]{r}{\hat{\tau}_f} \arrow[hook]{u}
			& \gr{K}_{\hat{f}} \arrow{r}{\hat{L}} \arrow[hook]{u}
			& \gr{K}_{\hat{g}} \arrow[hook]{u}
			& \gr{K}_g \arrow[dashed]{l}[swap]{\hat{\tau}_g} \arrow[hook]{u}
		\end{tikzcd}
	\end{center}
	Der alle vertikale avbildningane er den naturlege inklusjonen.
	Merk her at ettersom den naturlege inklusjonen er kontinuerlig, så er alle avbildningane på nederste rad også kontinuerlige frå at diagrammet kommuterar og definisjonen av underromstopologien. Meir spesifikt, så får ein at sidan $\tau_f$ og $\tau_g$ er homeomorfiar, at dei restrikterte avbildningane gitt av dei prikka linjene også er homeomorfiar.
	Og sidan $\hat{L}$ er både kontinuerlig og bijektiv, med invers $\tilde{L}$ som frå eit symmetrisk argument også er kontinuerlig, at $\hat{L}$ er ein homeomorfi.
	Så me får ein avbildning: $(\hat{\tau}_g)^{-1}\circ\hat{L}\circ\hat{\tau}_f:\gr{K}_f\to\gr{K}_g$ som er ein samansetting av tre homeomorfiar, og er derfor ein homeomorfi sjølve.
\end{proof}

\begin{remark}
	Grunna det førre resultatet så er det vanleg å snakka om \emph{den} geometriske realiseringa til eit abstrakt simplisielt kompleks $K$, ettersom alle forskjellige geometriske realiseringar er homeomorfe. Derfor plar ein ofte å sløyfa subskrifta i notasjona og kun bruka $\gr{K}$ for \emph{den} geometriske realiseringa til $K$.
\end{remark}

\begin{definition}
	La $\bar{B}_r(x):=\{y\in\Rb^m \mid d(x, y)\leq r\}$ vera den lukka ballen med hensyn til den euklidske metrikken i $\Rb^m$.
\end{definition}

\begin{definition}
	For $P$ ei endeleg mengda av punkt i $\Rb^m$ og $r\in[0,\infty)$, så er \emph{Cech-komplekset til $P$ med radius $r$} definert som:
	\[
		\Cech_r(P):=\left\{\sigma\subseteq P \mid \intersect_{p\in\sigma}\bar{B}_r(p)\neq\emptyset\right\}
	\]
\end{definition}

\begin{example}
	\phantom{abcd}
	\begin{figure}[htbp]
		\begin{center}
			\includesvg[width=0.8\textwidth]{Eksempel1_18-2.svg}
		\end{center}
		\caption{Her kan me sjå tre ulike eksempel av korleis Cech-komplekset ser ut med varierande radius.}
	\end{figure}
\end{example}

\begin{theorem} \label{thm:CASK}
	For $P$ ei endeleg mengda punkt i $\Rb^m$ og $r\in[0,\infty)$, så er Cech komplekset til $P$ med radius $r$ eit abstrakt simplisielt kompleks over $P$.
\end{theorem}

\begin{proof}
	For å visa dette, så må me visa at begge aksioma frå \autoref{def:ASK} held:
	\begin{enumerate}
		\item{ For ein \( \hat{p} \in P \) så ser me at for \( \sigma = \set{\hat{p}} \), så er \( \intersect_{p\in\sigma}\bar{B}_r(p)=\bar{B}_r(\hat{p})\neq\emptyset \) og då er \( \set{\hat{p}} \in \Cech_r(P) \) }
		\item{ Per definisjon av Cech-komplekset, så er alle \( \sigma \in \Cech_r(P) \) ei delmengda av \( P \) }
		\item{ For \( \sigma \in \Cech_r(P) \), så ser me at \( \intersect_{p\in\sigma} \bar{B}_r(p) \neq \emptyset \), men det betyr at for alle \( \tau \subseteq \sigma \), så er \( \intersect_{p\in\tau} \bar{B}_r(p) \) også ikkje-tom, fordi om den var tom, så ville: 
			\[ 
				\intersect_{p\in\sigma} \bar{B}_r(p) = \left( \intersect_{p\in(\sigma\setminus\tau)} \bar{B}_r(p) \right) \intersect \left( \intersect_{p\in\tau} \bar{B}_r(p) \right) = \left( \intersect_{p\in(\sigma\setminus\tau)} \bar{B}_r(p) \right) \intersect \emptyset = \emptyset 
			\] 
			som den ikkje er. Derfor må \( \tau \in \Cech_r(P) \) }
	\end{enumerate}
\end{proof}

Eit fint resultat som er ekvivalent til definisjonen av Cech komplekset:

\begin{theorem}
	For $P$ ei endeleg mengda av punkt i $\Rb^m$ og $r\in[0, \infty)$, så:
	\begin{equation*}
		\sigma\in \Cech_r(P) \Longleftrightarrow \exists x\in\Rb^m; \sigma \subseteq \bar{B}_r(x)
	\end{equation*}
\end{theorem}

\begin{proof}
	($\implies$)
	\\Sidan $A:=\intersect_{p\in\sigma}\bar{B}_r(p)\neq\emptyset$, så er det ein $x\in A$, der for $\forall p\in\sigma: d(p,x)\leq r$, sidan $x\in\bar{B}_r(p)$.
	Men sidan $d(p,x)=d(x,p)$ per definisjon av metrikk. 
	Så $\implies \forall p\in\sigma: p \in \bar{B}_r(x)$
	\\( \( \Longleftarrow \) )
	\\Ved eit likt argument så ser me at  $\forall p \in \sigma : d(x, p) \leq r \implies \forall p \in \sigma : d(p, x) \leq r \implies x \in \intersect_{p \in \sigma} \bar{B}_r(p) \implies \intersect_{p \in \sigma} \bar{B}_r(p) \neq \emptyset$
\end{proof}

Dette gir oss nok grunnlag til å endeleg forstå og definera nerva:

\begin{definition} % Kan definera for vilk. mengda? Og endeleg?
	La $X$ vera ei mengda. Vidare la $F$ vera ei mengda av delmengder av $X$. \emph{Nerva} til $F$, betegna som $\Nc(F)$, er:
	\begin{equation*}
		\Nc(F) := \left \{ \sigma \subseteq F \mid \intersect_{ F_i \in \sigma } F_i \neq \emptyset \right \}
	\end{equation*}
\end{definition}

\begin{theorem}
	For $X$ eit topologisk rom med $F$ ei mengda av delmengder av $X$, då er nerva til $F$ eit abstrakt simplisielt kompleks over $F$.
\end{theorem}

\begin{proof}
	Likt som i beviset for at Cech-komplekset var eit abstrakt simplisielt kompleks (\autoref{thm:CASK}) så må me visa alle betingelsane i \autoref{def:ASK}:
	\begin{enumerate}
		\item{ 
			For \( v \in F \) så har me at for \( \sigma = \set{ v } \) så er \( \intersect_{ p \in \sigma } p = v \neq \emptyset \), så \( \set{v} \in \Nc(F) \) 
		}
		\item{ 
			Alle element av \( \Nc(F) \) er per definisjon ei delmengda av \( F \)
		}
		\item{  
			For \( \sigma \in \Nc(F) \), så ser me at \( \intersect_{v\in\sigma} v \neq \emptyset \), men det betyr at for alle \( \tau \subseteq \sigma \), så er \( \intersect_{v\in\tau} v \) også ikkje-tom, fordi om den var tom, så ville: 
			\[ 
				\intersect_{v\in\sigma} v = \left( \intersect_{v\in(\sigma\setminus\tau)} v \right) \intersect \left( \intersect_{v\in\tau} v \right) = \left( \intersect_{v\in(\sigma\setminus\tau)} v \right) \intersect \emptyset = \emptyset 
			\] 
			som den ikkje er. Derfor må \( \tau \in \Nc(F) \)
		}

	\end{enumerate}
\end{proof}

\begin{example}
	\phantom{abc}
	\begin{figure}[htbp]
		\begin{center}
			\includesvg[width=0.7\textwidth]{Eksempel1_23.svg}
		\end{center}
		\caption{Her har me tre eksempel på mengdar av delmengder i \( \Rb^2 \) og den geometriske realiseringa av nerva deira.}
	\end{figure}
\end{example}

\begin{remark}
	Rett frå definisjonen så ser me at Cech komplekset til $P$ med radius $r$ er "<ekvivalent>" til nerva til $\union_{p \in P} \left \{ \bar{B}_r(p) \right \}$
\end{remark}

Då kan me endeleg utrykka nerveteoremet:

%\begin{definition}
%	La \( Y \) vera eit topologisk rom, og \( X \subseteq Y \) ei delmengda. Og la $A$ vera ei indekseringsmengda. Då er eit \emph{overdekke av $X$ relativ til $Y$} ei samling av delmengdar av $Y$: \( \{ F_\alpha \}_{\alpha\in A} \) der \( F_\alpha \subseteq Y  \) og \( X \subseteq \union_{\alpha \in A} F_\alpha \)
%\end{definition}

%\begin{theorem}
%	(Nerveteoremet for konveks og lukka overdekke) La $V$ vera eit vektorrom med ein topologi og \( F = \set{F_i}_{i=1}^n \) ei endeleg mengda av lukka og konvekse delmengder av $V$.
%	
%	Då er $\gr{\Nc(X)}$ homotopiekvivalent til \( \cup_{i=1}^n F_i \)
%\end{theorem}

%\begin{definition}
%	La \( X \) vera eit topologisk rom, og la \( F \) vera ei mengda av delmengder av \( X \). Då har \( F \) \emph{samantrekkbart snitt} om \( \forall \. \sigma \subseteq F \) så er \( \intersect_{F_i \in \sigma}F_i \) samantrekkbar.
%\end{definition}
%
%\begin{theorem}
%	(Opent, samantrekkbart snitt) La $V$ vera eit topologisk rom (ikkje nødvendigvis eit vektorrom), med \( X \subseteq V \) ei delmengda med underromstopologien. Vidare, la $F$ vera eit endeleg ope overdekke av $X$ relativ til \( X \) med samantrekkbart snitt.
%
%	Då er \( \gr{\Nc(F)} \) homotopiekvivalent til \( X \).
%\end{theorem}

\section{Bevis av Nerveteoremet}

Dette beviset er frå TODO

\begin{definition}
	Gitt eit endeleg abstrakt simplisielt kompleks \( K \) over \( V \), så er den \emph{barysentriske oppdelinga} av \( K \), betegna \( \Sd(K) \) alle tuplar: 
	\[
		\set{(\sigma_1, \sigma_2, \sigma_3, \dots, \sigma_n) \mid \sigma_1 \subset \sigma_2 \subset \sigma_3 \subset \dots \subset \sigma_n\,, \sigma_i \in K}
	\]
\end{definition}

\begin{lemma}
	Gitt eit endeleg abstrakt simplisielt kompleks \( K \) over \( V \), så er \( \Sd(K) \) eit endeleg abstrakt simplisielt kompleks over \( K \)
\end{lemma}

\begin{proof} \label{thm:subdivisjon-abstrakt-simplisielt-kompleks}
	TODO
\end{proof}

\begin{remark}
	Grunnen til at me betegnar den barysentriske oppdelinga til \( K \) for \( \Sd(K) \) er fordi den barysentriske oppdelinga av eit abstrakt simplisielt kompleks ofta kalla for ein "<subdivision>" på engelsk, og derfor brukar me \( \Sd \).
\end{remark}

\begin{lemma} \label{thm:geometrisk-kompleks-lukka}
	Geometrisk realisering lukka og kompakte
\end{lemma}

\begin{proof}
	TODO
\end{proof}

\begin{lemma} \label{thm:alpha-homeomorfi}
	For \( K \) eit endeleg abstrakt simplisielt kompleks over \( V \), så har me at:
	\( \gr{\Sd(K)}_f \subseteq \Rb^d \) er homeomorfi til \( \gr{K}_g \subseteq \Rb^m \) ved ein homeomorfi me kallar for \( \alpha \) som vert definert i slutten av beviset under.
\end{lemma}

\begin{proof} % Å herregud ditta beviset e vanskelegare enn eg trudde. % Shit, blande nokre bilete her. må dirrenetsiera mellom pre og post translasjon av mangder. Lineær operator definert av kor den sender n lineært uavhengige vektorar.
	Dette beviset bruker mykje den same strategien som i ref:TODO.

	Fyrst, lag ei ordning av elementa i \( V = (v_1, v_2, \dots, v_n) \) og ein ordning \( K = (\sigma_1, \sigma_2, \dots, \sigma_m) \) med \( \sigma_1 = \set{v_1} \). Vidare la \( \tau_f(x) := x-f(\set{v_1}) \), og la \( \tau_g(x) := x-g(v_1) \). Desse er homeomorfiar frå \autoref{thm:tau-homeomorfi}.

	Definer \( \hat{f} := \tau_f \circ f \) og \( \hat{g} := \tau_g \circ g \)

	Då kan me definera ein lineæroperator \( L: \Rb^d \to \Rb^m \) definert for \( \sigma \in K \) ved å senda \( \hat{f}(\sigma) \mapsto \frac{1}{\#\sigma}\sum_{v_j \in \sigma} \hat{g}(v_j) \). Dette er lov fordi \( \hat{f}(K) = (0, f(\sigma_2)-f(\sigma_1), f(\sigma_3)-f(\sigma_1), \dots, f(\sigma_n)-f(\sigma_1) ) \), er lineært uavhengige sidan \( f \) er ein affin imbedding, og \autoref{thm:geometrisklineærtuavhengig} (utanom \( 0 \), som blir sendt til \( 0 \) uansett frå linearitet), så \autoref{thm:definer-lin-op} gjeld.

	Me veit då frå \autoref{thm:begrensa-lin-op-er-kont} at \( L \) er kontinuerlig. 

	Vidare så får me at \( L(\gr{\Sd(K)}_{\hat{f}}) \subseteq \gr{K}_{\hat{g}} \) Fordi for \( x \in \gr{\Sd(K)}_{\hat{f}} \), så er \( x \) ein konveks kombinasjon av punkt \( \hat{f}(\sigma) \) for \( \sigma \in \Sd(K) \). Så er \( \sigma = (\sigma_1, \sigma_2, \dots, \sigma_{\#\sigma}) \). Då er \( x = \sum_{i=1}^{\#\sigma} a_i \hat{f}(\sigma_i) \), med \( (a_i)_i^{\#\sigma} \) dei barysentriske koordinatane til \( x \). Me får då:
	\begin{align*}
		L(x) &= L\left(\sum_{i=1}^{\#\sigma} a_i \hat{f}(\sigma_i)\right) \\
		&= \sum_{i=1}^{\#\sigma} L(a_i \hat{f}(\sigma_i)) \\
		&= \sum_{i=1}^{\#\sigma} a_i L(\hat{f}(\sigma_i)) \\
		&= \sum_{i=1}^{\#\sigma} a_i \frac{1}{\#\sigma_i} \sum_{v_j \in \sigma_i} \hat{g}(v_j) \\
		&= \sum_{i=1}^{\#\sigma} \sum_{v_j \in \sigma_i} a_i \frac{1}{\#\sigma_i} \hat{g}(v_j)
	\end{align*}
	Så gir dette at summen av koeffisientane til \( \hat{g}(v_j) \)-ane er:
	\[
		\sum_{i=1}^{\#\sigma} \sum_{v_j \in \sigma_i} a_i \frac{1}{\#\sigma_i} = 
		\sum_{i=1}^{\#\sigma} \#\sigma_i a_i \frac{1}{\#\sigma_i} =
		\sum_{i=1}^{\#\sigma} a_i = 1
	\]
	Og sidan for ein \( \sigma \in \Sd(K) \) så er \( \set{v_j \in V \mid v_j \in \sigma_i \,, \sigma_i \in \sigma} = \union_{\sigma_i \in \sigma} \sigma_i = \sigma_{\#\sigma} \in K \). Då er \( x \) i eit geometrisk simpleks utspunne av punkta \( \hat{g}(\sigma_{\#\sigma}) \), som er i \( \gr{K}_{\hat{g}} \)

	Definer \( \hat{L} := L|_{\gr{\Sd(K)}_{\hat{f}}}: \gr{\Sd(K)}_{\hat{f}} \to \gr{K}_{\hat{g}} \)

	Frå den universale eigenskapen av underromstopologien \autoref{thm:universal-eigenskap-underromstopologi} så får me at \( \hat{L} \) er kontinuerlig.

	Vidare, definer avbildninga \( \tilde{L} \) punktvis:
	
	For ein \( y \in \gr{K}_{\hat{g}} \), så er \( y \) ein konveks kombinasjon av punkt \( \hat{g}(\epsilon) \) for ein \( \epsilon \in K \), med barysentriske koordinatar \( \tuple{b_1, b_2, b_3, \dots, b_{\#\epsilon}} \). Velg ein ordning av \( \epsilon = \tuple{v_1, v_2, \dots, v_{\#\epsilon}} \) sånn at \( b_1 \geq b_2 \geq \dots \geq b_{\#\epsilon} \geq  b_{\epsilon+1}=0 \), og la
	\[
		\sigma := \tuple{\union_{j=1}^k\set{v_j}}_{k=1}^{\#\epsilon} = \tuple{\sigma_i}_{i=1}^{\#\epsilon} \in \Sd(K)
	\]
	% La
	% \begin{align*}
	% 	c &= \tuple{c_1, c_2, \dots, c_{\#\epsilon}} \\
	% 	&= \tuple{b_1-b_2, 2\tuple{b_2-b_3}, 3\tuple{b_3-b_2}, \dots, \#\epsilon b_{\#\epsilon}} \\
	% 	&= \tuple{i\tuple{b_i - b_{i+1}}}_{i=1}^{\#\epsilon}
	% \end{align*}
	Definer:
	\[
		\tilde{L}(y) = \tilde{L}\tuple{\sum_{i=1}^{\#\epsilon} b_i \hat{g}(v_i)} := \sum_{i=1}^{\#\epsilon}i\tuple{b_i-b_{i+1}}\hat{f}(\sigma_i)
	\]

	\( \tilde{L} \) er veldefinert med hensyn til ulike barysentriske koordinatar ettersom dei barysentriske koordinatane er unike frå \autoref{thm:unik-barysentrisk-koordinat}. I tilleg så er veldefinert med hensyn til ulik val av ordning, fordi om ein fikserar ein vilkårleg ordning på \( \epsilon \), så får ein kun forskjellige ordningar av dei barysentriske koordinatane. Og ein kan redusere alle permutasjonar av denne ordninga av dei barysentriske koordinatane til ein samansetting av fleire permutasjonar av kun to barysentriske koordinatar (ref:TODO?). Så anta me har to ordningar med \( n < m \):
	\[
		b_{i_1} \geq \dots \geq b_{i_n} \geq \dots \geq b_{i_m} \geq \dots \geq b_{i_{\#\epsilon}}
	\]
	Og
	\[
		b_{j_1} \geq \dots \geq b_{j_n} \geq \dots \geq b_{j_m} \geq \dots \geq b_{j_{\#\epsilon}}
	\]
	Der \( i_n=j_m \) og \( i_m=j_n \) har bytta posisjon, men resten er urørt. Då får me at \( b_{i_n} \geq b_{i_m}=b_{j_n} \geq b_{j_m}=b_{i_n} \) som betyr at \( b_{i_n} = b_{i_n+1} = \dots = b_{i_m}=b_{j_n}=b_{j_n+1}=\dots=b_{j_m} \). Og \( i_k = j_k \) for alle \( k \). Men det gir oss at \( c_{i_r} = c_{j_r} \) for alle \( r \). Og sidan alle moglege permutasjonar er ein samansetting av parvis permutasjonar som ikkje påverkar resultatet, så må \( \tilde{L} \) vera veldefinert, uavhengig av val av permutasjon av \( \epsilon \).

	Vidare så er \( \tilde{L}(\gr{K}_{\hat{g}}) \subseteq \gr{\Sd(K)}_{\hat{f}} \), fordi for \( y \in \gr{K}_{\hat{g}} \) så får ein:
	%fordi gitt \( y \in \gr{K}_{\hat{g}} \), ein konveks kombinasjon av \( \hat{g}(\epsilon) \) for ein \( \epsilon \in K \), med barysentriske koordinatar \( \tuple{b_1, b_2, b_3, \dots, b_{\#\epsilon}} \). Velg ein ordning av \( \epsilon = \tuple{v_1, v_2, \dots, v_{\#\epsilon}} \) sånn at \( b_1 \geq b_2 \geq \dots \geq b_{\#\epsilon} \geq  b_{\epsilon+1}=0 \)
	\[
		\tilde{L}(y) = \sum_{i=1}^{\#\epsilon}i\tuple{b_i-b_{i+1}}\hat{f}(\sigma_i)
	\]
	Og då er summen av koeffisientane:
	\begin{align*}
		\sum_{i=1}^{\#\epsilon}i\tuple{b_i-b_{i+1}} &= \sum_{i=1}^{\#\epsilon}\tuple{ib_i-ib_{i+1}} \\
		&= \sum_{i=1}^{\#\epsilon}i b_i - \sum_{i=1}^{\#\epsilon}i b_{i+1} \\
		&= \sum_{i=1}^{\#\epsilon}i b_i - \sum_{i=2}^{\#\epsilon}\tuple{i-1}b_{i} \\
		&= \sum_{i=1}^{\#\epsilon}i b_i - \sum_{i=2}^{\#\epsilon}i b_{i} + \sum_{i=2}^{\#\epsilon}b_i \\
		&= b_1 + \sum_{i=2}^{\#\epsilon} b_i \\
		&= \sum_{i=1}^{\#\epsilon} b_i \\
		&= 1
	\end{align*}
	Og sidan \( b_i \geq b_{i+1} \), så er \( i\tuple{b_i-b_{i+1}} \geq 0 \). Det vil seie, \( \tilde{L}(y) \) er ein konveks kombinasjon av den geometrisk uavhengige mengda \( \hat{f}(\sigma) \), for \( \sigma \in \Sd(K) \). Altså eit element i \( \gr{\Sd(K)}_{\hat{f}} \).

	Me kan derfor skriva \( \tilde{L}: \gr{K}_{\hat{g}} \to \gr{\Sd(K)}_{\hat{f}} \)

	\( \hat{L}\circ\tilde{L} = id \) fordi gitt ein \( y \in \gr{K}_{\hat{g}} \), så er det ein konveks kombinasjon av punkt \( \hat{g}(\epsilon) \) for \( \epsilon \in K \). Vel ein ordning av hjørnene: \( \epsilon = (v_1, v_2, \dots, v_{\#\epsilon}) \), og la \( y \) ha barysentrisk koordinatar \( (b_1, b_2, \dots, b_{\#\epsilon}) \)
	\[
		y = \sum_{i = 1}^{\#\epsilon} b_i \hat{g}(v_i)
	\]
	Velg ein ordning av hjørnene i \( \epsilon \) der \( b_1 \geq b_2 \geq \dots \geq b_{\#\epsilon} \) for \( y \). La 
	\begin{align*}
		(a_1, a_2, a_3, \dots, a_{\#\epsilon}) &= \left( b_1-b_2, 2(b_2-b_3), \dots, (\#\epsilon-1)(b_{\epsilon-1}-b_{\epsilon}), \#\epsilon b_{\epsilon} \right) \\
		&= \left( i (b_i-b_{i+1}) \right)_{i=1}^{\#\epsilon}, b_{\epsilon+1} = 0
	\end{align*}
		
	Og la \( \left(\union_{j = 1}^{k} \set{v_j} \right)_{k=1}^{\#\epsilon} = (\sigma_i)_{i=1}^{\#\epsilon}=\sigma \in \Sd(K) \)
	
	% Då definerar me: (Merk at \( \#\epsilon = \#\sigma\) )
	% \[
	% 	x = \sum_{i=1}^{\#\sigma} b_i \hat{f}(\sigma_i)
	% \]
	% Då er \( x \) ein konveks kombinasjon av punkta \( \hat{f}(\sigma) \), fordi:
	% \[
	% 	\sum_{i=1}^{\#\sigma} b_i = \sum_{i=1}^{\#\epsilon} a_i = 1
	% \]
	% Og sidan \( a_i \geq a_j \) for \( i \leq j \), så er alle \( b_i \geq 0 \). Derfor er \( x \in \gr{\Sd(K)}_{\hat{f}} \) % ref til def?
	% Om me anvender \( \hat{L} \) på denne \( y \)-en så får me:
	\begin{align*} % Meir detaljar rundt sum manipulasjon på linja 4-5?
		\hat{L}\circ\tilde{L}(y) &= \hat{L}\circ\tilde{L}\tuple{\sum_{i=1}^{\#\epsilon} b_i \hat{g}(v_i)} \\
		&= \hat{L}\left( \sum_{i=1}^{\#\epsilon} a_i \hat{f}(\sigma_i) \right) \\
		&= \sum_{i=1}^{\#\epsilon} \hat{L} \left( a_i \hat{f}(\sigma_i) \right) \\
		&= \sum_{i=1}^{\#\epsilon} a_i \hat{L} \left( \hat{f}(\sigma_i) \right) \\
		&= \sum_{i=1}^{\#\epsilon} a_i \frac{1}{\#\sigma_i} \sum_{v_j \in \sigma_i} \hat{g}(v_j) \\
		&= \sum_{i=1}^{\#\epsilon} a_i \frac{1}{i} \sum_{j = 1}^{i} \hat{g}(v_j) \\
		\intertext{Hugs \( b_{\epsilon+1} = 0 \)}
		&= \sum_{i=1}^{\#\epsilon} (b_{i}-b_{i+1}) \sum_{j = 1}^{i} \hat{g}(v_j) \\
		\intertext{Her gjer me eit ikkje-trivielt summasjonsskifte:}
		&= \sum_{j=1}^{\#\epsilon} \hat{g}(v_j) \sum_{i = j}^{\#\epsilon} (b_{i}-b_{i+1}) \\
		&= \sum_{j=1}^{\#\epsilon} b_i \hat{g}(v_j) \\
		&= y
	\end{align*}
	Ettersom \( y \) var eit vilkårleg element i \( \gr{K}_{\hat{g}} \), så er \( \hat{L}\circ\tilde{L} = id \).

	\( \tilde{L}\circ\hat{L} = id \) fordi gitt \( x \in \gr{\Sd(K)}_{\hat{f}} \) ein konveks kombinasjon av punkt \( \hat{f}(\hat{\sigma}) \) for \( \hat{\sigma} \in \Sd(K) \), med barysentriske koordinatar \( \tuple{\hat{a}_1, \hat{a}_2, \dots, \hat{a}_{\#\hat{\sigma}}} \). La \( n = \#\hat{\sigma}_{\hat{\sigma}} \) og la \( \sigma = \tuple{\sigma_i}_{i=1}^{n} := \tuple{v_j}_{j=1}^{n} \) for \( v_j \in \hat{\sigma}_{\hat{\sigma}} \). Me kan utvida \( \hat{\sigma} \) til \( \sigma \), ved å velga ein ordning av hjørnenene \( v_j \in \hat{\sigma}_{\hat{\sigma}} \) sånn at \( \hat{\sigma}_i = \sigma_{\#\hat{\sigma}_i} \). Det er mogleg fordi ein \( \hat{\sigma} \in \Sd(K) \) er ein strengt aukande følga av kompleks i \( K \), så ingen har likt antal element. I tillegg, så er \( \hat{\sigma}_i \subseteq \hat{\sigma}_{i+1} \) frå definisjonen av barysentrisk oppdeling. Merk at \( \sigma \in \Sd(K) \), ettersom det er ei streng følga av \( \sigma_i \in K \).

	Velg vidare barysentriske koordinatar til \( \sigma \):
	\[
		a_i =
		\begin{cases}
			a_i & \text{om \( i = \#\hat{\sigma}_j \) for ein eller annan \( j \)} \\
			0 & \text{ellers}
		\end{cases}
	\]
	Merk:
	\[
		\sum_{i=1}^n a_i = \sum_{i=1}^{\#\hat{\sigma}}\hat{a_i}=1
	\]
	Så me kan skriva \( x \) som ein konveks kombinasjon av element i \( \hat{f}(\sigma) \) med barysentriske koordinatar \( \tuple{a_i}_{i=1}^n \). Hugs at ettersom \( \hat{L} \) er veldefinert så er dette lov, og burde avbilde dette til same element.

	Då er:
	\begin{align*}
		\tilde{L}\circ\hat{L}(x) &= \tilde{L}\circ\hat{L}\tuple{\sum_{i=1}^n a_i \hat{f}(\sigma_i)} \\
		&= \tilde{L}\tuple{\sum_{i=1}^n a_i \hat{L}\tuple{\hat{f}(\sigma_i)}} \\
		&= \tilde{L}\tuple{\sum_{i=1}^n \frac{a_i}{\#\sigma_i}\sum_{v_j \in \sigma_i}\hat{g}(v_j)} \\
		&= \tilde{L}\tuple{\sum_{i=1}^n\sum_{v_j \in \sigma_i}\frac{a_i}{\#\sigma_i}\hat{g}(v_j)} \\
		\intertext{Eit ikkje-trivielt summasjonskifte:}
		&= \tilde{L}\tuple{\sum_{j=1}^n\sum_{\set{i:v_j\in\sigma_i}}\frac{a_i}{\#\sigma_i}\hat{g}(v_j)} \\
		&= \tilde{L}\tuple{\sum_{j=1}^n \hat{g}(v_j) \sum_{\set{i:v_j\in\sigma_i}}\frac{a_i}{\#\sigma_i}}
	\end{align*}
	La \( b_j = \sum_{\set{i:v_j\in\sigma_i}}\frac{a_i}{\#\sigma_i} \). Merk at \( \set{i : v_j \in \sigma_i} = \set{j, j+1, \dots, n } \) per definisjon av \( \sigma \). I tillegg, så ser ein at \( \#\sigma_i = i \), så me kan skriva: \( b_j = \sum_{i=j}^n \frac{a_i}{i} \).

	Men det er då tydleg at \( b_i \geq b_{i+j} \) ettersom den fyrste er den same summen som den siste, men med eit ekstra positivt ledd. Me kan derfor bruke denne ordninga i definisjonen av \( \tilde{L} \), og me kan derfor bruke den same definisjonen av \( \sigma \), så me får:
	\begin{align*}
		\tilde{L}\circ\hat{L}(x) &= \tilde{L}\tuple{\sum_{j=1}^n \hat{g}(v_j) \sum_{i:v_j\in\sigma_i}\frac{a_i}{\#\sigma_i}} \\
		&= \tilde{L}\tuple{\sum_{j=1}^n \hat{g}(v_j) \sum_{i=j}^n \frac{a_i}{i}} \\
		&= \sum_{j=1}^n j\tuple{\sum_{i=j}^n \frac{a_i}{i} - \sum_{i=j+1}^n \frac{a_i}{i}}\hat{f}(\sigma_j) \\
		&= \sum_{j=1}^n j\tuple{\frac{a_j}{j}}\hat{f}(\sigma_j) \\
		&= \sum_{j=1}^n a_j \hat{f}(\sigma_j) \\
		&= x
	\end{align*}
	Og ein får defor at \( \tilde{L}\circ\hat{L} = id \). 
	
	Dei to førre utsagna gir at \( \hat{L} \) er bijektiv, med invers \( \tilde{L} \).

	Vidare så er \( \hat{L} \) er ein lukka avbildning frå \autoref{thm:closed-map-lemma}, fordi det er ein kontinuerleg avbildning frå \( \gr{\Sd(K)}_{\hat{f}} \) til \( \gr{K}_{\hat{g}} \), der \( \gr{\Sd(K)}_{\hat{f}} \) er kompakt frå \autoref{thm:geometrisk-kompleks-lukka}, og  \( \gr{K}_{\hat{g}} \) er Hausdorff ettersom det er ei delmengda av \( \Rb^n \) for ein eller annan \( n \).

	Me får då frå TODO at sidan \( \hat{L} \) er ein bijektiv, kontinuerlig og lukka funksjon så er det ein homeomorfi.

	Me definerar \( \alpha: \gr{\Sd(K)}_f \to \gr{K}_g \):
	\[
		\alpha := \tau_g^{-1} \circ \hat{L} \circ \tau_f
	\]
	Og sidan \( \alpha \) er ein komposisjon av tre homeomorfiar, så er det ein homeomorfi sjølve.
\end{proof}

\begin{example}
	Teikn
\end{example}

\begin{lemma} \label{thm:konveks-kombinasjon-i-konveks}
	Alle konvekse kombinasjonar av (ikkje nødvendegvis geometrisk uavhengige) punkt \( P = \set{p_1, p_2, \dots, p_n } \subseteq V \), for \( V \) ei konveks mengda, er innehaldt i \( V \).
\end{lemma}

\begin{proof}
	Beviset er gjort ved induksjon på antal punkt i mengda punkt ein tek ein koveks kombinasjon over, her betegna \( n \).
	Grunntilfella:

	La \( n = 1 \):
	
	Då er \( x = a_1 p_1 \), med \( a_1 = 1 \), så \( x = p_1 \in K \).
	
	For \( n = 2 \):

	Då er \( x = a_1 p_1 + a_2 p_2 \), med \( a_1 + a_2 = 1 \). Men då er \( a_2 = 1 - a_1 \), så me kan skriva: \( x = a_1 p_1 + (1-a_1) p_2 \). Men dette er jo definisjonen på eit punkt på eit linjestykke mellom \( p_1 \) og \( p_2 \). Så frå definisjonen av ei konveks mengda, så må \( x \in K \).

	Annta det gjeld for \( n = k-1 \), vil no visa at det gjeld for \( n = k \).

	La \( x = \sum_{i=1}^k a_i p_i \). Me kan då skriva:
	\[ 
		x = a_1 p_1 + \sum_{i=2}^k a_i p_i = a_1 p_1 + (\sum_{i=2}^k a_i) \frac{\sum_{i=2}^k a_i p_i}{(\sum_{i=2}^k a_i)} = a_1 p_1 + (1-a_1) \frac{\sum_{i=2}^k a_i p_i}{(\sum_{i=2}^k a_i)}
	\]
	Men me ser at:
	\[
		\frac{\sum_{i=2}^k a_i p_i}{(\sum_{i=2}^k a_i)} = \sum_{i=2}^k p_i\frac{a_i}{(\sum_{i=2}^k a_i)}
	\]
	Dannar ein konveks kombinasjon av \( k-1 \) punkt sidan:
	\[
		\sum_{i=2}^k \frac{a_i}{(\sum_{i=2}^k a_i)} = \frac{\sum_{i=2}^k a_i}{(\sum_{i=2}^k a_i)} = 1
	\]
	Og
	\[
		\frac{a_i}{(\sum_{i=2}^k a_i)} \geq 0, \, i=2,3,\dots,k
	\]
	Så:
	\[
		y := \frac{\sum_{i=2}^k a_i p_i}{(\sum_{i=2}^k a_i)}
	\]
	Er eit element i \( V \) av induksjonshypotesen. Derfor er:
	\[
		x = a_1 p_1 + \sum_{i=2}^k a_i p_i = a_1 p_1 + (1-a_1) y
	\]
	Ein konveks kombinasjon av \( 2 \) punkt i \( V \), og me får frå induksjonshypotesen at \( x \in V \).
\end{proof}

\begin{definition} \label{thm:Gamma} % Kan fjerna phi og f avhengigheita frå notasjonen. U, phi, f avhengigheita er implisitt. Burde ha med affine imbeddinga?
	Gitt \( U = \set{U_i}_{i=1}^n \) ei endeleg mengda av konvekse mengder i \( \Rb^m \). Vidare, for kvar \( \epsilon \in \Nc(U) \) velg punkt \( v_\epsilon \in \intersect_{u \in \sigma} u \).

	For \( x \in \gr{\Sd(\Nc(U))}_f \) ein konveks kombinasjon av dei geometrisk uavhengige punkta \( f(\sigma) \) for ein eller annan \( \sigma \in \Sd(\Nc(U)) \), der \( \sigma = \tuple{\sigma_1, \sigma_2, \dots, \sigma_{\#\sigma}} \) med dei barysentriske koordinatane \( \tuple{a_1, a_2, \dots, a_{\#\sigma}} \), definer:
	\[
		\Gamma(x) = \sum_{i=1}^{\#\sigma} a_i v_{\sigma_i}
	\]
\end{definition}

\begin{lemma} \label{thm:Gamma-eigenskapar} % Må visa veldefinert mhp f og phi?
	\( \Gamma \) frå \autoref{thm:Gamma} er veldefinert, kontinuerleg og biletet er i \( \union_{u\in U} u \).
\end{lemma}

\begin{proof}
	%\( \Gamma \) er veldefinert ettersom dei barysentriske koordinatane den er definert ut ifrå er unike frå \autoref{thm:unik-barysentrisk-koordinat}.

	\( \Gamma \) er kontinuerleg og veldefinert fordi om ein let \( \hat{\tau} : \gr{\Sd(\Nc(U))}_f \to \gr{\Sd(\Nc(U))}_{\hat{\tau} \circ f} \) vera definert som i \autoref{thm:tau-homeomorfi} (\( x \mapsto x - f(\hat{\sigma}) \) for ein \( \hat{\sigma} \in \Nc(U) \)), så er dette ein homeomorfi, og om ein deretter definerar \( L \) til å vere ein lineæravblildning som tek \( \hat{\tau} \circ f (\sigma) \mapsto v_{\sigma} \) for alle \( \sigma \in \Nc(U) \). Og vidare let \( \hat{L} := L|_{\gr{\Sd(\Nc(U))}_{\hat{\tau}\circ f}} \), så ser me at for \( x \in \gr{\Sd(\Nc(U))}_f \) ein konveks kombinasjon av dei geometrisk uavhengige punkta \( f(\sigma) \) for ein eller annan \( \sigma \in \Sd(\Nc(U)) \), der \( \sigma = \tuple{\sigma_1, \sigma_2, \dots, \sigma_{\#\sigma}} \) med dei barysentriske koordinatane \( \tuple{a_1, a_2, \dots, a_{\#\sigma}} \) så er:
	\begin{align*}
		\hat{L} \circ \hat{\tau} (x) &= \hat{L} \circ \hat{\tau} \tuple{\sum_{i=1}^{\#\sigma}a_i f(\sigma_i)} \\
		&= \hat{L} \tuple{\sum_{i=1}^{\#\sigma}a_i f(\sigma_i)-f(\hat{\sigma})} \\
		&= \hat{L} \tuple{\sum_{i=1}^{\#\sigma}a_i f(\sigma_i)-\tuple{\sum_{j=1}^{\#\sigma}a_i}f(\hat{\sigma})} \\
		&= \hat{L} \tuple{\sum_{i=1}^{\#\sigma}a_i f(\sigma_i)-\tuple{\sum_{j=1}^{\#\sigma}a_if(\hat{\sigma})}} \\
		&= \hat{L} \tuple{\sum_{i=1}^{\#\sigma}a_i\tuple{f(\sigma_i)-f(\hat{\sigma})}} \\
		&= \hat{L} \tuple{\sum_{i=1}^{\#\sigma}a_i\hat{\tau}\tuple{f(\sigma_i)}} \\
		&= \sum_{i=1}^{\#\sigma}a_iL\tuple{\hat{\tau}\tuple{f(\sigma_i)}} \\
		&= \sum_{i=1}^{\#\sigma}a_i v_{\sigma_i} \\
		&= \Gamma(x)
	\end{align*}
	Og sidan \( \hat{\tau} \) er kontinuerleg og veldefinert, og \( \hat{L} \) er kontinuerleg og veldefinert frå \autoref{thm:begrensa-lin-op-er-kont} og \autoref{thm:universal-eigenskap-underromstopologi}, så må \( \Gamma = \hat{L}\circ\hat{\tau} \) også vera kontinuelreg og veldefinert.

	Biletet til \( \Gamma \) er i \( \union_{u\in U} u \), fordi for \( x \in \gr{\Sd(\Nc(U))}_f \) ein konveks kombinasjon av dei geometrisk uavhengige punkta \( f(\sigma) \) for ein eller annan \( \sigma \in \Sd(\Nc(U)) \), der \( \sigma = \tuple{\sigma_1, \sigma_2, \dots, \sigma_{\#\sigma}} \) med dei barysentriske koordinatane \( \tuple{a_1, a_2, \dots, a_{\#\sigma}} \), så får me at:
	\[
		\Gamma(x) = \sum_{i=1}^{\#\sigma} a_i v_{\sigma_i}
	\]
	Er ein konveks kombinasjon av punkta \( v_{\sigma_i} \) for \( i = 1,2,\dots,\#\sigma \). Men sidan \( \sigma_1 \subsetneq \sigma_2 \subsetneq \dots \subsetneq \sigma_{\#\sigma} \), så er det ein \( \epsilon \in U \) med \( \epsilon \in \sigma_1, \sigma_2, \dots, \sigma_{\#\sigma} \). Og derfor så er \( v_{\sigma_1}, v_{\sigma_2}, \dots, v_{\sigma_{\#\sigma}} \in \epsilon \). Men ettersom alle elementa i \( U \) er konvekse mengder, så er \( \Gamma(x) \) ein konveks kombinasjon av element i ei konveks mengda, \( \epsilon \), og frå \autoref{thm:konveks-kombinasjon-i-konveks}, så er \( x \in \epsilon \subseteq \union_{u\in U} u \).
\end{proof}

\begin{example}
	Teikn Gamma TODO
\end{example}

\begin{definition} % Definér open ball?
	For ei mengda \( U_i \) og ein \( \epsilon > 0 \) definer \( U_i^\epsilon := \union_{x \in U_i} B_\epsilon(x) \), og for ei endeleg mengda av mengder \( U=\set{U_i}_{i=1}^n \) definer \( U^\epsilon := \set{U_i^\epsilon}_{i=1}^n \).
\end{definition}

\begin{lemma} \label{thm:epsilondekke} % Kunne ha skrive det som at f danner ein bijeksjon av simpleksar?
	For alle endelege mengder av kompakte, delmengder av \( \Rb^d \), \( U = \set{U_i}_{i=1}^n \), så \( \exists \epsilon > 0 \) sånn at \(f \) som tek 
	\[ 
		\set{U_{i_1}, U_{i_2}, \dots, U_{i_k}} \mapsto \set{U_{i_1}^\epsilon, U_{i_2}^\epsilon, \dots, U_{i_k}^\epsilon} 
	\] 
	dannar ein bijeksjon frå  \(\Nc(U) \) til \( \Nc(U^\epsilon) \).
\end{lemma}

\begin{proof}
	La \( A \) vere mengda av alle mogelge delmengder av \( U \), og \( A^\epsilon \) vere alle moglege delmengder av \( U^\epsilon \). Då ser me at \( f \) dannar ein bijeksjon frå \( A \) til \( A^\epsilon \). Og sidan \( i: \Nc(U) \hookrightarrow A \), inklusjonsavbildninga, er injektiv, så er \( f \circ i \) injektiv.

	Ettersom \( U^{\epsilon} \) kan kun få fleire snitt, enn \( U \), så må \( f(\Nc(U)) \subseteq \Nc(U^{\epsilon}) \).
	
	For å visa \( f(\Nc(U)) \supseteq \Nc(U^{\epsilon}) \), så ser me på det kontrapostive tilfellet:
	 
	\[ 
		f(\Nc(U)) \supseteq \Nc(U^{\epsilon}) \iff \tuple{\forall \sigma \in \Nc(U^{\epsilon}) \implies \sigma \in f(\Nc(U))} \iff \tuple{\forall \sigma \not\in f(\Nc(U)) \implies \sigma \not\in \Nc(U^{\epsilon})}
	\] % Finskriva med A og A^\epsilon?

	For ein \( \sigma = \set{U_{i_1}, U_{i_2}, \dots, U_{i_k}} \in A \) så kan me skriva det som \( \sigma_J \) for \( J = \set{i_1, i_2, \dots i_k} \). Likt for \( \sigma^\epsilon_J \in A^\epsilon \). Så me vil visa at \( \exists \epsilon : \sigma^\epsilon_J \not\in f(\Nc(U)) \implies \sigma^\epsilon_J \not\in \Nc(U^{\epsilon}) \) for alle moglege \( J \) der \( \sigma^\epsilon_J \not\in f(\Nc(U)) \). Men ettersom \( f \) dannar ein bijeksjon frå \( A \) til \( A^\epsilon \), så kan me sjå på \( \exists \epsilon : \sigma_J \not\in \Nc(U) \implies f(\sigma_J) \not\in \Nc(U^{\epsilon}) \).

	La:
	\[
		r_J := sup_{x \in \union_{j \in J} u_j} d(x,0)
	\]
	Sidan \( \union_{j \in J} u_j \) er ein endeleg union av kompakte mengder så er den kompakt frå \autoref{thm:endeleg-union-kompakt-er-kompakt}. Då får ein frå \autoref{thm:supremum-over-kompakt-er-maks}, at \( r_J = \max{x \in \union_{j \in J} u_j} d(x,0) \) og derfor endeleg. Me får også at \( \union_{j \in J} u_j \subseteq \bar{B}_{r_J}(0) \).
	
	Definer vidare \( D_J := \bar{B}_{4r_J}(0) \). Sidan dette er ein lukka og begrensa mengda i \( \Rb^d \), så er den kompakt frå \autoref{thm:heine-borel}.

	La \( g_J: D_J \to \Rb \) med \( x \mapsto \max_{j \in J} d(x, U_j) \). Då ser ein at \( g \) er kontinuerleg sidan det er ein maks av endeleg mange kontinuerlege funskjonar frå \autoref{thm:distanse-er-kont} og \autoref{thm:maksimum-av-kont-er-kont}.

	Om ein let \( \sigma_J \not\in \Nc(U) \), så får ein at \( g_J(x) > 0 \) for alle \( x \) i \( D_J \), sidan \( \intersect_{u \in \sigma_J} u = \emptyset \).

	Og sidan \( D_J \) er kompakt. Då får me frå \autoref{thm:infimum-over-kompakt-er-min}, at \( g_J(x) \) oppnår eit minimum, som frå argumentet over må vere ulik \( 0 \). Dette kallar me for
	\[
		\epsilon_J := \min_{x \in D_J} \max_{j \in J} d(x, U_j) 
	\]
	Men dette minimumet er eit globalt minimum over heile \( \Rb^d \), fordi for \( x \in \Rb^d \setminus D_J \), så er \( g_J(x) \geq 3r_J \), sidan det er minste moglege avstand frå \( x \) til ei av \( u_j \in U_J \). Men om \( x \in \bar{B}_{r_J}(0) \) då er \( g_J(x) \leq 2r_J \) sidan det er det lengste vekke den andre mengda kan vere og framleis vere inne i \( \bar{B}_{r_J}(0) \). Så eit globalt minimum må vere eit minimum inne i \( D_J \).

	Vidare så ser ein at \( \sigma_J^{\epsilon_J} \not\in \Nc(U^{\epsilon_J}) \) fordi om ein let \( x \) vere eit av punkta som gav det globale minimumet \( \epsilon_J \), så er \( x \not\in \intersect_{j \in J} u_j^{\epsilon_J} \) fordi det er ingen \( y \in  u_{\hat{j}} \) for \( \hat{j} \) den verdien i \( J \) som gav maksimumet til \( \max_{j \in J} d(x, U_j) \), som har \( d(y, x) > \epsilon_J \). Så det er ingen open ball om \( y \) med radius \( r_J \) som inneheld \( x \). Derfor kan ikkje \( x \) vere i snittet av alle \( u_j \)-ane. Det same argumentet gjeld for alle andre punkt i \( D_J \) som ikkje var globale minimum.
	
	Så det er ingen punkt i \( \intersect_{j \in J} u_j^{\epsilon_J} \), og derfor er \( \sigma_J^{\epsilon_J} \not\in \Nc(U^{\epsilon_J}) \).

	Vidare, la: 
	\[
		\epsilon := \min_{\set{J : \sigma_J \not\in \Nc(U)}} \epsilon_J
	\]
	\( \epsilon \neq 0 \) fordi alle \( \epsilon_J \neq 0 \) og det er kun endeleg mange \( \sigma_J \in A \).

	Om me brukar den øvre \( \epsilon \)-en så får me at for alle \( \sigma_J \not\in \Nc(U) \), så er \( \sigma_J^{\epsilon}=f(\sigma_J) \not\in \Nc(U^{\epsilon}) \). Som var det me ville visa.

\end{proof}

\begin{remark}
	Avbildninga f over blir ofta kalla ein simplisiell isomorfi, og er det eg alluderte til når eg sa at nerva var ekvivalent med cech komplekset i ref TODO
\end{remark}

\begin{definition} \label{def:psi} % Kva er betingelsane på overdekket her?
	For ei endeleg mengda av kompakte delmengder \( U_i \subseteq \Rb^d \), med \( U = \set{U_i}_{i=1}^n \), velg ein \( \epsilon \) som i \autoref{thm:epsilondekke}, kan då danna:
	\[
		\phi_i(x) := \frac{d(x, \Rb^m \setminus U_i^\epsilon)}{d(x, U_i) + d(x, \Rb^m \setminus U_i^\epsilon)}
	\]
	Og vidare:
	\[
		\psi_i(x) := \frac{\phi_i(x)}{\sum_{k=1}^n \phi_k(x)}
	\]
\end{definition}

\begin{lemma}
	\( \phi_i \) og \( \psi_i \) frå \autoref{def:psi} er kontinuerlege på domenet \( \union_{i=1}^n U_i^\epsilon \).
\end{lemma}

% Burde refera til eit lemma om sum av kont avb er kont
\begin{proof} 
	Sidan distansefunksjonen \( d \) er kontinuerleg frå \autoref{thm:distanse-er-kont}, og nemnaren er alltid ulik null sidan \( U_i \subsetneq U_i^\epsilon \) og ein sum av to kontinuerlege funskjonar og derfor kontinuerkeg sjølve, så er \( \phi_i \) ein kvotient av to kontinuerlege funskjonar med heile \( \Rb^d \) som domene, og derfor kontinerleg på heile \( \union_{i=1}^n U_i^\epsilon \) sjølve.

	Ved mykje det same argumentet så ser me at \( \psi_i \) er ein kvotient med ein kontinuerleg avbildning i teljaren og ein sum av kontinerlege avbildningar som er alltid ulik null på \( \union_{i=1}^n U_i^\epsilon \) sidan minst ein \( \phi_i \) er ulik null fordi \( phi_i(x)=0 \) kun når \( d(x, \Rb^m \setminus U_i^\epsilon)=0 \) som skjer når \( x \not\in U_i^\epsilon \). Men sidan for \( x \in \union_{i=1}^n U_i^\epsilon \) så er det vertfall ein \( U_i^\epsilon \) som \( x \) er eit element av. Derfor er \( \psi_i \) kontinerleg.
\end{proof}

% \begin{lemma} % Burde bruka alle
% 	For ei endeleg mengda av lukka, kompakte delmengder \( U_i \subseteq \Rb^d \), med \( U = \set{U_i}_{i=1}^n \) , la \( \psi_i \) og \( \epsilon \) vera definert som i \autoref{thm:psi}. Då er følgjande sant:
% 	\begin{enumerate}
% 		\item{For \( x \in  U_i \), då er \( \psi_i(x) \geq \psi_j(x) \forall j \) }
% 		\item{For \( x \in \Rb^d \setminus U_i^\epsilon \), då er \( \psi_i(x)=0 \) }
% 		\item{For \( x \in \union_{u \in U} u \), så er \( \sum_{i=1}^n \psi_i(x) = 1 \) }
% 		\item{For \( x \in \union_{u \in U} u \), så er \( \psi_i(x) \geq 0 \) }
% 	\end{enumerate}
% \end{lemma}

% \begin{proof}
% 	TODO
% \end{proof}

\begin{definition} \label{def:Psi}
	For ei endeleg mengda av lukka, kompakte delmengder \( U_i \subseteq \Rb^d \), med \( U = \set{U_i}_{i=1}^n \). Vidare la \( \psi_i \) vera som i \autoref{def:psi}, og \( \gr{\Nc(U)}_f \) ein geometrisk realisering av nerva til \( U \) med hensyn til den affine imbeddinga \( f \). Definer for \( x \in \union_{u \in U} u \):
	\[
		\Psi(x) := \sum_{i=1}^n \psi_i(x)f(U_i)
	\]
\end{definition}

\begin{lemma} \label{thm:psi-kont} % Noko meir?
	\( \Psi \) frå \autoref{def:Psi} er kontinuerlig, og har bilete i \( \gr{\Nc(U)}_f \).
\end{lemma}

\begin{proof} % Trenge fleire forkunskapar om kvifor ting er kontinuerleg
	Sidan alle \( \psi_i \)-ane er kontinuerlege, så er \( \psi_i(x) f(U_i) \) også kontinuerleg, ettersom \( f(U_i) \) er ein konstant vektor. Så \( \Psi \) er ein sum av kontinuerlege funksjonar og derfor kontinuerleg sjølve.

	For å visa at \( \Psi\tuple{\union_{u \in U} u} \subseteq \gr{\Nc(U)}_f \), la \( x \in \union_{u \in U} u \). Då er \( \Psi(x) = \sum_{i=1}^n \psi_i(x)f(U_i) \). Men merk at:
	\begin{align*}
		\sum_{j=1}^n \psi_j(x)  &= \sum_{j=1}^n \frac{\phi_j(x)}{\sum_{k=1}^n \phi_k(x)} \\
		&= \frac{\sum_{j=1}^n \phi_j(x)}{\sum_{k=1}^n \phi_k(x)} \\
		&= 1
	\end{align*}
	Og sidan \( \phi_i(x) \geq 0 \) for alle \( i \) og \( x \) så er \( \psi_i(x) \geq 0 \) for alle \( i \) og \( x \). Derfor så får me at \( \Psi(x) \) er ein konveks kombinasjon av punkta \( f(U) \).

	Ettersom \( x \not\in U_i^\epsilon \iff \phi_i(x) = 0 \iff \psi_i(x) = 0 \). Så får me at \( \psi_i(x) \neq 0 \iff x \in U_i^\epsilon \). Så la \( A_x^\epsilon := \set{U_i^\epsilon \in U^\epsilon : x \in U_i^\epsilon} \). Då ser me at \( A_x^\epsilon \in \Nc(U^\epsilon) \). Men med \( h(x) \) bijeksjonen me får frå \autoref{thm:epsilondekke}, så veit me at \( A_x^\epsilon \) korrsponderar til ein \( A_x := h^{-1}(A_x^\epsilon) \in \Nc(U) \).
	
	Så for alle \( x \in \union_{u \in U} u \) så er \( \Psi(x) \) ein konveks kombinasjon av punkta \( f(A_x) \) om ein forkastar alle hjørnene med barysentrisk koordinat lik null. Dette er i \( \gr{\Nc(U)}_f \).
\end{proof}

\begin{definition} \label{def:bst}
	For \( K \) eit endeleg abstrakt simplisielt kompleks over \( V \) med \( v \in V \), så er \emph{den lukka barysentriske stjerna} til \( K \) i \( v \), betegna \( \bst(v) \) lik:
	\[
		\bst(v) = \set{\sigma \in \Sd(K) \mid \sigma \union \set{v} \in \Sd(K)}
	\]
\end{definition}

\begin{example}
	TODO
\end{example}

\begin{lemma} \label{thm:bst-ask}
	Den lukka barysentriske stjerna som definert i \autoref{def:bst} er eit abstrakt simplisielt kompleks, med hjørnemengda \( W = \set{w \in K : \set{w} \union \set{v} \in \Sd(K)} \)
\end{lemma}

\begin{proof}
	Her må me visa dei tre eigenskapane frå \autoref{def:ASK}:
	\begin{enumerate}
		\item{For \( w \in W \), så er \( \set{w} \in \bst(v) \) frå definisjonen.}
  		\item{For \( \sigma \in \bst(v) \) så er \( \sigma \subseteq W \), fordi for ein vilkårleg \( \tau \in \sigma \), så er \( \sigma \union \set{v} = \set{\tau} \union \sigma \union \set{v} \in \Sd(K) \). Men sidan \( \Sd(K) \) er eit abstrakt simplisielt kompleks frå \autoref{thm:subdivisjon-abstrakt-simplisielt-kompleks}, så er \( \sigma \union \set{v} \supseteq \set{\tau} \union \set{v} \in \Sd(K) \). Og då er \( \tau \in W \).}
    	\item{For \( \sigma in \bst(v) \), la \( \tau \subseteq \sigma \). Sidan \( \sigma \union \set{v} \in \Sd(K) \), og sidan \( \Sd(K) \) er eit abstrakt simplisielt kompleks frå \autoref{thm:subdivisjon-abstrakt-simplisielt-kompleks}, så er \( \sigma \union \set{v} \supseteq \tau \union \set{v} \in \Sd(K) \). Og derfor er \( \tau \in \bst(v) \).}
	\end{enumerate}
\end{proof}

\begin{lemma} \label{thm:Gamma-inni-ui}
	Gitt \( U = \set{U_i}_{i=1}^n \) ei endeleg mengda av konvekse mengder i \( \Rb^m \). La \( \Gamma \) vere definert som i \autoref{thm:Gamma}.

	Då er \( \Gamma(\gr{\bst(U_i)}_f) \subseteq U_i \)
\end{lemma}

\begin{proof}
	Frå \autoref{thm:bst-ask}, så veit me at hjørnemengda til \( \bst(U_i) \), er
	\[
		W = \set{w \in \Nc(U) : \set{w} \union \set{U_i} \in \Sd(\Nc(U))}=\set{w \in \Nc(U) : U_i \in w}
	\] 
	Men då er jo \( v_w \) (frå \autoref{thm:Gamma}) \( \in \intersect_{u \in w} u \subseteq U_i \) for alle \( w \). Og sidan \( \Gamma \) tek eit element, \( x \), i \( \gr{\bst(U_i)}_f \) som er ein konveks kombinasjon av punkta \( \set{f(w)}_{w \in W} \) til ein konveks kombinasjon av \( \set{v_w}_{w \in W} \subseteq U_i \), så får me at \( \Gamma(x) \) er ein konveks kombinasjon av punkta \( \set{v_w}_{w \in W} \), som alle ligg i ei konveks mengda \( U_i \), og me får derfor frå \autoref{thm:konveks-kombinasjon-i-konveks} at \( x \in U_i \).
\end{proof}

\begin{lemma} \label{thm:bst-betingingar}
	La \( U = \tuple{U_i}_{i=1}^n \) vera ei endeleg mengda av delmengder av \( \Rb^d \). La \( \alpha \) vera definert som i \autoref{thm:alpha-homeomorfi} der \( \alpha: \gr{\Sd(\Nc(U))}_f \to \gr{\Nc(U)}_g \). For ein \( x \in \gr{\Nc(U)}_g \), ein konveks kombinasjon av punkta \( g(\tau) \) for ein eller annan \( \tau \in \Nc(U) \) med barysentriske koordinatar \( \tuple{b_1, b_2,\dots, b_i, \dots, b_{\#\tau}} \) med \( b_i \geq b_j, \, \forall j \in \set{1, 2, 3, \dots, \#\tau} \). Då er:
	\[ 
		\alpha^{-1}(x) \in \gr{\bst(U_i)}
	\]
\end{lemma}

\begin{proof}
	Fyrst merk at
	\[
		\alpha^{-1} = \tau_f^{-1} \circ \tilde{L} \circ \tau_g
	\]
	Så me får:
	\begin{align*}
		\alpha^{-1}(x) &= \tau_f^{-1} \circ \tilde{L} \circ \tau_g (x) \\
		&= \tau_f^{-1} \circ \tilde{L} \circ \tau_g \tuple{\sum_{j=1}^{\#\tau} b_j g(U_j)} \\
		\intertext{Sidan \( \tau_g \), bevarer barysentriske koordinatar frå beviset til \autoref{thm:tau-homeomorfi}:}
		&= \tau_f^{-1} \circ \tilde{L} \tuple{\sum_{j=1}^{\#\tau} b_j \tau_g \circ g(U_j)} \\
	\end{align*}
	Men frå antaginga så har me at \( b_i \geq b_j \, \forall j \in \set{1, 2, 3, \dots, \#\tau} \). Så me kan velga ein ordning av \( U : \tuple{U_{j_i}}_{i=1}^n \). Sånn at: \( b_{j_1} \geq b_{j_2} \geq b_{j_3} \geq \dots \geq b_{j_{\#\tau}} \). Og \( b_{j_1}=b_i \). Og me har då 
	\[
		\sigma = \tuple{\sigma_i}_{i=1}^{\#\tau} = \tuple{\union_{i=1}^k \set{U_{j_i}}}_{k=1}^{\#\tau} 
	\]
	La også: \( a_k = k\tuple{b_{j_k}-b_{j_{k+1}}} \)

	Så får ein at:
	\begin{align*}
		\alpha^{-1}(x) &= \tau_f^{-1} \tuple{\sum_{l=1}^{\#\tau}a_k \tau_f \circ f(\sigma_i)} \\
		&= \sum_{l=1}^{\#\tau} a_k f(\sigma_i)
	\end{align*}
	Men sidan \( \sigma \in \bst(U_i) \) sidan \( \sigma \union \set{U_i} = \sigma \in \Sd(\Nc(U)) \), og \( \sum_{k=1}^{\#\tau} a_k = 1 \), og \( a_k \geq 0, \, \forall k \) frå beviset til \autoref{thm:alpha-homeomorfi}. Så er \( \alpha(x) \) ein konveks kombinasjon av punkta i \( f(\sigma), \sigma \in \bst(U_i) \), og derfor:
	\[
		\alpha^{-1}(x) \in \gr{\bst{U_i}}_f
	\]
\end{proof}

\begin{lemma} \label{thm:Psi-inni-bst}
	For ei endeleg mengda av kompakte delmengder \( U_i \subseteq \Rb^d \), med \( U = \set{U_i}_{i=1}^n \). Vidare la \( \Psi \) vera som i \autoref{def:Psi}, og \( \gr{\Sd(\Nc(U))}_f \) ein geometrisk realisering av den barysentriske oppdelinga til nerva av \( U \) med hensyn til den affine imbeddinga \( f \). La \( \alpha \) vera som i \autoref{thm:alpha-homeomorfi}

	Då er \( \alpha \circ \Psi(U_i) \subseteq \gr{\bst(U_i)}_f \)
\end{lemma}

\begin{proof}
	La \( x \in U_i \). Då ser me at:
	\[
		\phi_i(x) := \frac{d(x, \Rb^m \setminus U_i^\epsilon)}{d(x, U_i) + d(x, \Rb^m \setminus U_i^\epsilon)} = \frac{d(x, \Rb^m \setminus U_i^\epsilon)}{d(x, \Rb^m \setminus U_i^\epsilon)} = 1
	\] 
	Og for \( j \neq i \), sidan \( d(x, U_j) \geq 0 \) så får ein:
	\[
		\phi_j(x) := \frac{d(x, \Rb^m \setminus U_j^\epsilon)}{d(x, U_j) + d(x, \Rb^m \setminus U_j^\epsilon)} \leq \frac{d(x, \Rb^m \setminus U_j^\epsilon)}{d(x, \Rb^m \setminus U_j^\epsilon)} = 1
	\]
	Så \( \phi_i(x) \geq \phi_j(x) \) for alle \( j \).

	Dette gir oss at:
	\[
		\psi_i(x) = \frac{\phi_i(x)}{\sum_{k=1}^n \phi_k(x)} \geq \frac{\phi_j(x)}{\sum_{k=1}^n \phi_k(x)} = \psi_j(x)
	\]
	Og sidan \( \Psi(x) \) er ein konveks kombinasjon av punkta \( f(\tau) \) for \( \tau = \set{U_i : \psi_i(x) \neq 0} \in \Nc(U) \), med barysentriske koordinatar \( A:= \tuple{\psi_j(x)}_{\set{ j : U_j\in \tau}} \) frå beviset til \autoref{thm:psi-kont}. Og \( \phi_i(x) \neq 0 \), så er \( U_i \in \tau \), og \( \phi_i(x) \geq \phi_j(x) \) for alle \( \phi_j(x) \in A \). Så frå \autoref{thm:bst-betingingar}, så er:
	\[
		\alpha \circ \Psi(x) \in \gr{\bst(U_i)}_f
	\]
	Sidan dette var uavhengig av val av \( x \in U_i \) så får ein:
	\[
		\alpha \circ \Psi(U_i) \subseteq \gr{\bst(U_i)}_f
	\]
\end{proof}

\begin{lemma}
	La \( U = \tuple{U_i}_{i=1}^n \) vera ei endeleg mengda av kompakte og konvekse delmengder av \( \Rb^d \). La \( \Gamma \) vera som i \autoref{thm:Gamma}, la \( \Psi \) vera som i \autoref{def:Psi}, og la \( \alpha \) vera som i \autoref{thm:alpha-homeomorfi}. Då er:
	\[
		\Gamma \circ \alpha \circ \Psi \simeq id_{\union_{u \in U} u} 
	\]
\end{lemma}

\begin{proof} % Koffor er homotopien kontinuerlig? TODO
	For \( x \in U_i \) så følger det frå \autoref{thm:Psi-inni-bst} at:
	\[
		\alpha \circ \Psi(x) \in \gr{\bst(U_i)}_f
	\]
	Og frå \autoref{thm:Gamma-inni-ui} så får ein derfor at:
	\[
		\Gamma \circ \alpha \circ \Psi(x) \in U_i
	\]
	Sjå på avbildninga:
	\[
		H: \union_{u \in U} u \times I \to \union_{u \in U} u
	\]
	som tek:
	\[
		\tuple{x, t} \mapsto xt + \Gamma \circ \alpha \circ \Psi(x)(1-t)
	\]
	Denne er kontinuerlig fordi \( \Gamma \) er kontinuerlig frå \autoref{thm:Gamma-eigenskapar}, \( \Psi \) er kontinuerlig frå \autoref{thm:psi-kont}, og \( \alpha \) er ein homeomorfi.
	
	Den er også veldefinert fordi om ein fikserar ein \( x \in U_i \) så er \( xt + \Gamma \circ \alpha \circ \Psi(x)(1-t) \in U_i \) for alle \( t \in I \) sidan det er på på ei linja mellom to punkt i \( U_i \) som er konveks.

	Så \( H \) er ein homotopi frå \( id \) til \( \Gamma \circ \alpha \circ \Psi \)
\end{proof}

\begin{definition}
	For \( K \) eit endeleg abstrakt simplisielt kompleks over hjørnemengda \( V \), så er \emph{\( n \)-skjelettet} til \( K \), betegna \( \Sk_n(K) \) lik:
	\[
		\Sk_n(K) := \set{\sigma \in K : \#\sigma \leq n}
	\]
\end{definition}

\begin{lemma}
	For \( K \) eit endeleg abstrakt simplisielt kompleks over hjørnemengda \( V \), så er \( n \)-skjelettet til \( K \) eit abstrakt simplisielt kompleks.
\end{lemma}

\begin{proof}
	Alle \( \tau \in \Sk_n(K) \) er delmengder av ein \( \sigma \in K \), så dei oppfyllar krav 2 og 3. Krav 1 er oppfylt ved at \( \Sk_0(K) \subset \Sk_n(K) \).
\end{proof}

\begin{definition}
	For \( K \) eit endeleg abstrakt simplisielt kompleks over hjørnemengda \( V \), for \( \sigma = \tuple{\sigma_i}_{i=1}^{\#\sigma} \in K \), så er \emph{grensa til \( \sigma \)}, betegna \( \partial\sigma \) lik:
	\[
		\partial\sigma := \union_{i = 1}^{\#\sigma} \sigma \setminus \sigma_i 
	\]
\end{definition}

\begin{lemma}
	For \( K \) eit endeleg abstrakt simplisielt kompleks over hjørnemengda \( V \), og \( \gr{K}_f \) ei geometrisk realisering av \( K \). For \( \sigma \in K \), la \( G_f(\sigma) \) betegna det geometriske simplekset utspunne av punkta \( f(\sigma) \), og \( G_f(\partial\sigma) := \union_{\tau \in \partial\sigma} G_f(\tau) \). Då er:
	\[
		G_f(\partial\sigma) \times I \union G_f(\sigma) \times \set{0, 1}
	\] 
	Homeomorf til \( S^{\#\sigma} \), Og
	\[
		G_f(\sigma) \times I
	\] 
	Er homeomorf til \( D^{\#\sigma+1} \).
\end{lemma}

\begin{proof}
	Hakkje peiling korleis eg skal visa ditta...
\end{proof}

\begin{lemma}
	bst er eit good cover TODO
\end{lemma}

\begin{lemma}
	kan utvida avb frå Sn til samantrekkbar til Dn+1 til samantrekkbar. TODO
\end{lemma}

\begin{lemma} % Jævli vanskeleg å visa frå botnen
	La \( U = \tuple{U_i}_{i=1}^n \) vera ei endeleg mengda av kompakte og konvekse delmengder av \( \Rb^d \). La \( \Gamma \) vera som i \autoref{thm:Gamma}, la \( \Psi \) vera som i \autoref{def:Psi}, og la \( \alpha \) vera som i \autoref{thm:alpha-homeomorfi}. Då er:
	\[
		\alpha^{-1} \circ \Psi \circ \Gamma \simeq Id_{\gr{\Sd(\Nc(U))}_f}
	\]
\end{lemma}

\begin{proof}
	Dette beviset er gjort ved induksjon over \( n \)-skjelettet til \( \Sd(\Nc(U)) \) for å finna ein \( H_n : \Sk_n(\Sd(\Nc(U))) \times I \to \Sk_n(\Sd(\Nc(U))) \) som sender \( \Sk_n(\bst(U_i)) \to \bst(U_i) \)
\end{proof}

% \begin{lemma} % Går ikkje... ;_;
% 	La \( U = \tuple{U_i}_{i=1}^n \) vera ei endeleg mengda av kompakte og konvekse delmengder av \( \Rb^d \). La \( \Gamma \) vera som i \autoref{thm:Gamma}, la \( \Psi \) vera som i \autoref{def:Psi}, og la \( \alpha \) vera som i \autoref{thm:alpha-homeomorfi}. Då er:
% 	\[
% 		\Psi \circ \Gamma \circ \alpha^{-1} \simeq Id_{\gr{\Nc(U)}_g}
% 	\]
% \end{lemma}

% \begin{proof}
% 	La \( y \in \alpha(\bst(U_i)) \) for ein vilkårleg \( i \). Me ser at for \( x \) ein konveks kombinasjon av hjørnemengda til \( \bst(U_i) \), som me kallar for \( W \), frå \autoref{thm:bst-ask} med barysentriske koordinatar \( \tuple{a_1, a_2, \dots, } \)
% \end{proof}

\begin{theorem} %Nerveteoremet for endeleg, reel, konveks, kompakt overdekke
	Tek spesiell overdekke
	Gamma psi er homotopiekvivalen
	psi gamma
\end{theorem}

\begin{proof}
	TODO
\end{proof}

\section{Anvendingar av Nerveteoremet}

\end{document}
