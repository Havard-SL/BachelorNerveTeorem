\documentclass[a4paper, titlepage, 12pt, norsk]{article}

\usepackage[utf8]{inputenc} % For å kunna skriva æøå i tekstar
\usepackage[T1]{fontenc}
\usepackage[margin=2cm]{geometry} % Fiksa margin
\usepackage[ddmmyyyy]{datetime} % Fiksar datorormatet på tiitelen
\usepackage{amssymb}
\usepackage{amsmath} % For visse mattesymbol, typ \mathbb
\usepackage{graphicx} % Bilete
%\usepackage{minted} % For kodesnuttar og resultat
\usepackage{enumitem} % Kan endra på korleis listar ser ut

\usepackage[hidelinks,colorlinks=true]{hyperref} % For autoref
\hypersetup{allcolors=[rgb]{0,0.31,0.62}} % For fargar på ting ein referer til i autoref

\usepackage{amsthm} % For teorem, definisjon, bevis enviornments.
\usepackage{thmtools} 

\usepackage{svg}
\svgpath{svg/}
\usepackage{float}
%\usepackage[subsection]{placeins}

%\usepackage{tikz}
%\usetikzlibrary{matrix}
%\newcommand{\diagram}[3]{\matrix (#1) [matrix of math nodes,row
%  sep={#2},column sep={#3},text height=1.5ex,text
%  depth=0.25ex]}

\usepackage{tikz-cd} %For kommutative diagram med tikz
\usepackage{parskip} %Fjernar indents, men gjer linjeavstanden kortare

\usepackage{fontspec}
\usepackage{unicode-math}

%\setmainfont{Atkinson Hyperlegible}
\setmainfont{Fira Sans}
\setmathfont{Fira Math}
\setmathfont[range=\setminus]{Asana Math}

% Ny type lista med ganske perfekt spacing
\newlist{plist}{enumerate}{5}
\setlist[plist]{align=left, itemindent = 0cm, labelsep = 0cm, labelindent = 0cm}
\setlist[plist,1]{label=\arabic*, font=\bf\Large}
\setlist[plist,2]{label*=.\arabic*, labelwidth=1.25cm, leftmargin=1.25cm}
\setlist[plist,3]{label*=.\arabic*, labelwidth=1.5cm, leftmargin=1.5cm}

\theoremstyle{plain}
\newtheorem{theorem}{Teorem}[section]
\newtheorem{proposition}[theorem]{Proposjon}
\newtheorem{corollary}[theorem]{Korollar}
\newtheorem{lemma}[theorem]{Lemma}

\theoremstyle{definition}
\newtheorem{definition}[theorem]{Definisjon}
\newtheorem{example}[theorem]{Eksempel}
\newtheorem{remark}[theorem]{Merknad}

\usepackage[norsk]{babel}
%\renewcommand*{\proofname}{Bevis}

\newcommand{\Rb}{\mathbb{R}}
\newcommand{\Qb}{\mathbb{Q}}
\newcommand{\Zb}{\mathbb{Z}}
\newcommand{\Nb}{\mathbb{N}}
\newcommand{\Nc}{\mathcal{N}}
%\newcommand{\Cech}{ \operatorname{Cech} }
\DeclareMathOperator{\Cech}{Cech}
\newcommand{\intersect}{ \mathop{\cap}\limits } % For finare grenser på snitt
\newcommand{\union}{ \mathop{\cup}\limits }
\newcommand{\gr}[1]{ \lvert #1 \rvert } % Geometrisk realisering
\newcommand{\set}[1]{ \left \{ #1 \right \} } % Mengda
\DeclareMathOperator{\Sd}{Sd}

% PERSONLEGE NOTAT

% Større separator i mengdar
% Rydda vekk semikolonar og kolonar i teksten min
% \cup og \cap ser merkelege ut
% Bytta font? Litt uleseleg til tider. Ingen kaligrafi N, \bigcup/\bigcap f.eks
% Fiksa tikz diagrammet
% Skriva om det longe beviset
% Fleire eksempel!!
% Kva er feil med \vert og |?
% Nervedefinisjon for topologisk rom og kun endeleg?
% Hugs å spesifisér alt! Skriv ned idéen bak korleis eg skriv ting.
% Seksjon 0 med definisjonar og minimumskrav?
% Figurtekst skal vera kort, eksempelteksten skal referera til figurteksten
% Eksempeltekst til kvar figur / figur må bli nemd minst ein gong.
% Skriv om beviset for 1.17 (homeomorfi mellom geometriske realiseringar.)
% Legg til intuitiv forklaring på kva som foregår i byrjinga.

% PLAN
% Fokus på å forbetra del 1, og så byrja på del 2. Pass på Magnus sitt bevis er litt shady.

% MERK:
% Drit i uendeleg!


\title{Nerveteorement og Anvendingar}
\author{Håvard Skjetne Lilleheie}

\begin{document}

\maketitle

\section{Grunnleggande definisjonar og resultat}

\begin{definition} \label{def:ASK}
	For $V$ ei ikkje-tom og endeleg mengda, så definerar me $K$ til å vera eit \emph{endeleg asbtrakt simplisielt kompleks over $V$} om:
	\begin{enumerate}
		\item{$\forall v \in V: \{v\} \in K$}
		\item{\( \forall \sigma \in K: \sigma \subseteq V \)}
		\item{$\forall \sigma \in K, \forall \tau \subseteq \sigma: \tau \in K$}
	\end{enumerate}
	Då kallar me og $V$ for \emph{hjørnemengda} til $K$.
\end{definition}

Nokre eksempel av endelege abstrakte simplisielle kompleks over $\{p_1, p_2, p_3\}$:

\begin{itemize}
	\item{$A_1=\{\{p_1, p_2\}, \{p_1\}, \{p_2\}\}$}
	\item{$A_2=\{\{p_1\}, \{p_2\}, \{p_3\}, \{p_1, p_3\}, \{p_2, p_3\}\}$}
	\item{$A_3=\{\{p_1\}, \{p_2\}, \{p_3\}\}$}
\end{itemize}

Og her er nokre ikkje-eksempel av endelege abstrakte simplisielle kompleks over \( \{p_1, p_2, p_3\} \):

\begin{itemize}
	\item{$A_1'=\{\{p_1, p_2\}, \{p_1\}\}=A_1 \setminus \{\{p_2\}\}$}
	\item{$A_2'=\{\{p_1\}, \{p_2\}, \{p_3\}, \{p_1, p_3\}, \{p_2, p_3\}, \{p_1, p_2, p_3\}\}=A_2 \union \{\{p_1, p_2, p_3\}\}$}
	\item{$A_3'=\{\{p_1\}, \{p_2\}, \{p_2, p_3\}\}=\left(A_3 \setminus \{\{p_3\}\}\right) \union \{\{p_2, p_3\}\}$}
\end{itemize}

Inspirasjonen bak denne definisjonen kan verka tilfeldig og heilt umotivert, men som me kjem til å sjå seinare; så er dette ein svært praktisk simplifisering av informasjon som bevarer mykje av eigenskapane til den originale datamengda ein vil sjå på.

Fyrst, lat oss sjå korleis dette kan tenkast på som meir geometrisk:

\begin{definition}
	Ein endeleg mengda punkt $\{p_1, p_2, p_3, \dots, p_n\}$ i $\Rb^m$ for $m\geq1$ er \emph{geometrisk uavhengige} om for $\{a_i\}_{i=0}^n$ med $a_i\in\Rb$, så:
	\begin{equation*}
		\left(\sum_{i=1}^n a_i=0 \land  \sum_{i=1}^n a_ip_i=0\right)\implies a_i=0 \; \forall i
	\end{equation*}
\end{definition}

\begin{theorem}\label{thm:geometrisklineærtuavhengig}
	Ei endeleg mengda $\{p_1, p_2, p_3, \dots, p_n \}$ av punkt i $\Rb^m$ er gemetrisk uavhengige $\Longleftrightarrow$ vektorane $\{(p_2-p_1), (p_3-p_1), (p_4-p_1),\dots,(p_n-p_1)\}$ er lineært uavhengige i $\Rb^m$
\end{theorem}

\begin{proof}
	($\implies$)
	\\For $a_i\in\Rb$, annta $\sum_{i=2}^na_i(p_i-p_1)=0$. Om me då definerar: $a_1 := -\sum_{i=2}^na_i$, så ser me at 
	\begin{equation*}
		\sum_{i=1}^na_i=\sum_{i=2}^na_i-\sum_{i=2}^na_i=0
	\end{equation*}
	og at 
	\begin{equation*}
		\sum_{i=1}^na_ip_i=\sum_{i=2}^na_i(p_i-p_1)=0
	\end{equation*}
	Og sidan $\{p_i\}_{i=1}^n$ er geometrisk uavhengig frå antaginga, så $\implies$ $a_i=0 \, \forall i$. Som er definisjonen på lineært uavhengig.
	\\(\(\Longleftarrow\))
	\\for $a_i\in\Rb$, annta $\sum_{i=1}^n a_i=0$ og $\sum_{i=1}^n a_ip_i=0$. Då ser me at 
	\begin{equation*}
		a_1=-\sum_{i=2}^n a_i
	\end{equation*} 
	Det gir oss: 
	\begin{equation*}
		0=\sum_{i=1}^n a_ip_i=\sum_{i=2}^n a_ip_i-\sum_{i=2}a_ip_1=\sum_{i=2}a_i(p_i-p_1)
	\end{equation*}
	Men sidan $\{(p_i-p_1)\}_{i=2}^n$ er lineært uavhengig, så: $\implies a_i = 0$ for $i\in[2,n]$. Men sidan $a_1 = -\sum_{i=2}^n a_i=0$, så får me $a_i=0 \forall i$. Som er definisjonen på geometrisk uavhengig.
\end{proof}

\begin{remark}
	Ein direkte konsekvens frå dette resultatet og grunnleggande lineær algebra er at ein geometrisk uavhengig mengda i $\Rb^m$ kan maksimalt innehalda $m+1$ forskjellige punkt. Dette er fordi det kan ikkje vera meir enn $m$ lineært uavhengige vektorar i eit $m$-dimensjonalt vektorrom.
\end{remark}

\begin{definition}
	\sloppy Ein \emph{konveks kombinasjon} av ei geometrisk uavhengig mengda $P=\{p_1, p_2, p_3, \dots, p_n\}$ av punkt i $\Rb^m$ er eit punkt $x\in\Rb^m$ gitt ein tupel av koeffisientar $A=\{a_1, a_2, a_3, \dots, a_n\}$ i $\Rb$, med $a_i\geq0\forall i$ og $\sum_{i=1}^n a_i = 1;$
	
	\begin{equation*}
		x=\sum_{i=1}^n a_ip_i
	\end{equation*}
	Om $P$ er ein tupel (ein ordna mengda) så blir koeffiseintane også ein tupel, og $A$ blir då kalla dei \emph{barysentriske koordinatane} til $x$.
\end{definition}

\begin{theorem}
	Dei barysentriske koordinatane til ein konveks kombinasjon er eintydig.
\end{theorem}
\begin{proof}
	La $x$ vera ein konveks kombinasjon av $(p_1, p_2, p_3, \dots, p_n)$ i $\Rb^m$. Annta at $x$ har to barysentriske koordinatar: $(a_1, a_2, a_3, \dots, a_n)$ og $(b_1, b_2, b_3, \dots, b_n)$. Det betyr at:
	\begin{equation*}
		0 = x - x = \sum_{i=1}^n a_ip_i - \sum_{i=1}^n b_ip_i=\sum_{i=1}^n (a_i-b_i)p_i
	\end{equation*}
	og me får:
	\begin{equation*}
		\sum_{i=1}^n(a_i-b_i)=\sum_{i=1}^na_i - \sum_{i=1}^nb_i = 1 - 1 = 0
	\end{equation*}
	Men sidan $(p_1, p_2, p_3, \dots, p_n)$ er geometrisk uavhengig, så 
	\begin{equation*}
		\implies (a_i-b_i)=0\forall i \Longleftrightarrow a_i = b_i \forall i
	\end{equation*}
	og dei barysentriske koordinatane er derfor like.
\end{proof}

\begin{definition}
	Det \emph{geometrisk simplekset} utspent av ei geometrisk uavhengig mengda av punkt $P=\{p_1, p_2, p_3, \dots, p_n\}$ i $\Rb^m$, er alle konvekse kombinasjonar av $P$.
	\\Vidare, så kallar me dette geometriske simplekset for eit \emph{$(n-1)$-simpleks}, når $P$ består av $n$ punkt.
\end{definition}

\begin{remark}
	Ut ifrå den førre definisjonen så ser me at eit $0$-simpleks kun er eit enkelt punkt, eit $1$-simpleks er alle konvekse kombinasjonar mellom to punkt, som viser seg å vera ei linjestykket mellom dei to punkta. Og eit $2$-simpleks dannar ein trekant. Ein $3$-simpleks blir også kalla eit tetraheder. Denne simpleks definisjonenen gir ein fin matematisk forklaring av "trekant"-strukturar i $\Rb^m$.
\end{remark}

\begin{example}
	\phantom{123}
	\begin{figure}[htbp]
		\begin{center}
			\includesvg[width=0.6\textwidth]{Eksempel1-8.svg}
			\caption{Her er tre geometriske simpleksar. Ein \( 1 \)-simpleks utspunne av \( \set{a_1, a_2} \), Ein \(2\)-simpleks utspunne av \( \set{b_1, b_2, b_3} \), og ein \(3\)-simpleks utspunne av \( \set{c_1, c_2, c_3, c_4} \)}
		\end{center}
	\end{figure}
\end{example}

\begin{remark}
	Noko som er svært interessant med denne definisjonen er at om ein har ei geometrisk uavhengig mengda $P$, og ser på $\hat{P}_i := P \setminus \{p_i\}$, så vil dette også vera ei geometrisk uavhengig mengda. I tilleg så vil det geometriske simplekset utspent av $p^i$ vere ei delmengda av det geometriske simplekset utspent av $P$. Med andre ord; alle delmengdar av $P$ dannar også andre geometriske simpleks, inneholdt i det originale geometriske simplekset! Men ikkje nok med det, fordi når ein ser på det geometriske simplekset utspent av $\hat{P}_i$ for ein eller annan vilkårleg $i$, så ser me at det er som eine "fjeset" av det geometriske simplekset utspunne av $P$. Dette inspirerar ein ny definisjon:
\end{remark}

\begin{definition}
	Eit \emph{fjes} til eit geometrisk simpleks utspunne av $P=\{p_1, p_2, p_3, \dots, p_n\}$ er eit geometriske simpleks utspunne av $\hat{P}_i := P\setminus \{p_i\}$ for ein eller annan $i$.
\end{definition}

\begin{definition} % Ikkje utspunne av nokre punkt?
	Eit \emph{geometrisk simplisielt kompleks} er ei mengda av geometriske simpleksar (utspent av punkt i $\Rb^m$), $K$, sånn at:
	\begin{enumerate}
		\item{For $\sigma \in K$ så er alle fjesa til $\sigma$ også i $K$.}
		\item{For $\sigma, \tau \in K$ med $\sigma \intersect \tau \neq \emptyset$, då er $\sigma \intersect \tau$ eit fjes av både $\sigma$ og $\tau$.}
	\end{enumerate}
\end{definition}

\begin{theorem}
	TODO Geometrisk simplisielle kompleks har ein utvida unik barysentrisk koordinat.
\end{theorem}

\begin{definition}
	For $V$ ein endeleg mengda og $m\geq1$, så er $f:V\rightarrow \Rb^m$ ein \emph{affin imbedding} om $f$ er injektiv, og om biletet, $f(A)$ er ei geometrisk uavhengig mengda av punkt.
\end{definition}

\begin{definition} % Må ha ein affin imbedding for alt? Sjekk opp i ditta TODO
	For eit endeleg abstrakt simplisielt kompleks $K$, over ei ikkje-tom hjørnemengda $V$, gitt ein vilkårleg affin imbedding $f:V\to\Rb^m$ for ein eller annan $m\geq1$. Så er den \emph{geometriske realiseringa med hensyn til $f$} unionen av alle dei geometriske simpleksane utspunne av punkta i $f(\sigma)$, for $\sigma\in K$. Dette er ofte betegna $\gr{K}_f$
\end{definition}

\begin{theorem}
	TODO geometrisk realisering er eit geometrisk simplisielt kompleks
\end{theorem}

\begin{theorem}
	Geometrisk realisering er eintydig opp til homeomorfi. Med andre ord: For to ulike geometriske realiseringar av eit endeleg abstrakt simplisielt kompleks $K$, med hensyn til $f$ og $g$ (to ulike affine imbeddingar), så er $\gr{K}_f$ og $\gr{K}_g$ homeomorfe.
\end{theorem}

\begin{proof}%Rotete? Trenge unike barysentriske koordinatar? Må visa \hat{\tau} er bijeksjonar
	La $K$ vera eit abstrakt simplisielt kompleks, og la $f:K\to\Rb^m$ og $g:K\to\Rb^l$ vera to affine imbeddingar. La $V=\{ v_1, v_2, \dots, v_n \}$ vera hjørnemengda til $K$ Definér vidare $x_i=(f(k_{i-1})-f(k_1))$ og $y_i=(g(k_{i-1})-g(k_1))$. La $\tau_f:\Rb^m\to\Rb^m$ vera ein forskyving som tek $x\mapsto x-f(k_1)$. Og la $\tau_g:\Rb^l\to\Rb^l$ vera ein forskyving som tek $x\mapsto x-g(k_1)$. Til slutt, la $L_1:\Rb^m\to\Rb^l$ vera den lineære funksjonen som sender $x_i\mapsto y_i$.
	Og for praktiske grunnar: Definér $\hat{f}:=\tau_f\circ f$ og $\hat{g}:=\tau_g \circ g$
Me får då at $L_1(\hat{f}(k_i))=\hat{g}(k_i)$, fordi:
For $i=1$:

	\begin{equation*}
		L_1(\hat{f}(k_1)=L_1(f(k_1)-f(k_1))=L(0)=0=g(k_1)-g(k_1)=\hat{g}(k_1)
	\end{equation*}

	For $i\neq 1$:

	\begin{equation*}
		L_1(\hat{f}(k_i))=L_1(f(k_i)-f(k_1))=L_1(x_{i-1})=y_{i-1}=g(k_i)-g(k_1)=\hat{g}(k_i)
	\end{equation*}
	Om ein då let $\hat{L}=L|_{\gr{K}_{\hat{f}}}$, og $L_2:\Rb^l\to\Rb^m$ vera den lineære funskjonen som sender $y_i\mapsto x_i$ og let $\tilde{L}=L_2|_{\gr{K}_{\hat{g}}}$, så er definisjonane symmetriske og det symmetriske resultatet gjelder derfor for $L_2$ også. 
	For eit vilkårleg element $x\in\gr{K}_{\hat{f}}$ så er $x$ eit element av ein simpleks i det geometriske simplisielle komplekset. Det vil seie, me kan utrykka $x$ som ein konveks sum av ein geometrisk uavhengig mengda, som korresponderar til den simpleksen $x$ er eit element i.
	
	Me velger derfor ein $\sigma\in K$ sånn at $\hat{f}(\sigma)=\{\hat{f}(k_1), \hat{f}(k_2), \dots, \hat{f}(k_r)\}$. Dette er ein geometrisk uavhengig mengda fordi $\tau_f$ er injektiv og sender $x_i\mapsto x_i$ som er lineært uavhengig og biletet er derfor geometrisk uavhengig om $\{x_i\}$ er lineært uavhengig, som det er frå \autoref{thm:geometrisklineærtuavhengig}.
	
	La denne mengda utspenna punkta til ein simpleks som $x$ er eit element av. Då kan me utrykka $x$ som ein konveks kombinasjon av elementa i $\hat{f}(\sigma)$: $x=\sum_{i=1}^ka_i\hat{f}(k_i)$ med $\sum_{i=1}^ka_i=1$ og $a_i\geq0\; \forall i$. 
	
	Her er valet av kva geometrisk simpleks me velger ikkje vikitg, ettersom $\hat{L}$ er veldefinert.
	\begin{align*}
		\tilde{L}\circ \hat{L}(x) &= \tilde{L}\circ \hat{L}\left(\sum_{i=1}^ra_i\hat{f}(k_i)\right) \\
		&= \tilde{L}\left(\sum_{i=1}^r\hat{L}(a_i\hat{f}(k_i))\right) \\
		&= \tilde{L}\left(\sum_{i=1}^ra_i\hat{L}(\hat{f}(k_i))\right) \\
		&= \tilde{L}\left(\sum_{i=1}^ra_i\hat{g}(k_i))\right) \\
		\intertext{Sidan $\sum_{i=1}^ra_i\hat{g}(k_i)\in\gr{K}_{\hat{g}}$, så får me:} \\
		&= \sum_{i=1}^r\tilde{L}(a_i\hat{g}(k_i)) \\
		&= \sum_{i=1}^ra_i\tilde{L}(\hat{g}(k_i)) \\
		&= \sum_{i=1}^ra_i\hat{f}(k_i) \\
		&= x
	\end{align*}
	Og likt for $\hat{L}\circ\tilde{L}(y)$.
	Så det betyr at $\hat{L}$ er bijektiv med invers $\tilde{L}$.
	Om ein let $\hat{\tau}_f:=\tau_f|_{\gr{K}_f}$ og $\hat{\tau}_g:=\tau_g|_{\gr{K}_g}$, så kan ein sjå at $\hat{\tau}_f\gr{K}_f\to\gr{K}_{\hat{f}}$ er ein bijeksjon, og likt for $\hat{\tau}_g$.
	\\Då får ein følgande kommutative diagram:
%	\begin{center}
%		\begin{tikzpicture}
%			\diagram{d}{2.5em}{2.5em}{
%				\Rb^m & \Rb^m & \Rb^l & \Rb^l \\
%				\vert K\vert_f & \vert K\vert_{\hat{f}} & \vert K\vert_{\hat{g}} & \vert K\vert_g \\
%				};
%		\path[->,font = \scriptsize, midway]
%		(d-1-1) edge node[above]{$\tau_f$} (d-1-2)
%		(d-1-2) edge node[above]{$L$} (d-1-3)
%		(d-1-4) edge node[above]{$\tau_g$} (d-1-3)
%		(d-2-1) edge [dashed,->] node[below]{$\hat{\tau}_f$}(d-2-2)
%		(d-2-2) edge node[below]{$\hat{L}$} (d-2-3)
%		(d-2-4) edge [dashed,->] node[below]{$\hat{\tau}_g$} (d-2-3)
%		(d-2-1) edge (d-1-1)
%		(d-2-2) edge (d-1-2)
%		(d-2-3) edge (d-1-3)
%		(d-2-4) edge (d-1-4);
%		\end{tikzpicture}
%	\end{center}
	\begin{center} % |, \vert funke ikkje??
		\begin{tikzcd}
			\Rb^m \arrow{r}{\tau_f}
			& \Rb^m \arrow{r}{L}
			& \Rb^l
			& \Rb^l \arrow{l}[swap]{\tau_g} \\
			\gr{K}_f \arrow[dashed]{r}{\hat{\tau}_f} \arrow[hook]{u}
			& \gr{K}_{\hat{f}} \arrow{r}{\hat{L}} \arrow[hook]{u}
			& \gr{K}_{\hat{g}} \arrow[hook]{u}
			& \gr{K}_g \arrow[dashed]{l}[swap]{\hat{\tau}_g} \arrow[hook]{u}
		\end{tikzcd}
	\end{center}
	Der alle vertikale avbildningane er den naturlege inklusjonen.
	Merk her at ettersom den naturlege inklusjonen er kontinuerlig, så er alle avbildningane på nederste rad også kontinuerlige frå at diagrammet kommuterar og definisjonen av underromstopologien. Meir spesifikt, så får ein at sidan $\tau_f$ og $\tau_g$ er homeomorfiar, at dei restrikterte avbildningane gitt av dei prikka linjene også er homeomorfiar.
	Og sidan $\hat{L}$ er både kontinuerlig og bijektiv, med invers $\tilde{L}$ som frå eit symmetrisk argument også er kontinuerlig, at $\hat{L}$ er ein homeomorfi.
	Så me får ein avbildning: $(\hat{\tau}_g)^{-1}\circ\hat{L}\circ\hat{\tau}_f:\gr{K}_f\to\gr{K}_g$ som er ein samansetting av tre homeomorfiar, og er derfor ein homeomorfi sjølve.
\end{proof}

\begin{remark}
	Grunna det førre resultatet så er det vanleg å snakka om \emph{den} geometriske realiseringa til eit abstrakt simplisielt kompleks $K$, ettersom alle forskjellige geometriske realiseringar er homeomorfe. Derfor plar ein ofte å sløyfa subskrifta i notasjona og kun bruka $\gr{K}$ for \emph{den} geometriske realiseringa til $K$.
\end{remark}

\begin{definition}
	La $\bar{B}_r(x):=\{y\in\Rb^m \mid d(x, y)\leq r\}$ vera den lukka ballen med hensyn til den euklidske metrikken i $\Rb^m$.
\end{definition}

\begin{definition}
	For $P$ ei endeleg mengda av punkt i $\Rb^m$ og $r\in[0,\infty)$, så er \emph{Cech-komplekset til $P$ med radius $r$} definert som:
	\[
		\Cech_r(P):=\left\{\sigma\subseteq P \mid \intersect_{p\in\sigma}\bar{B}_r(p)\neq\emptyset\right\}
	\]
\end{definition}

\begin{example}
	\phantom{abcd}
	\begin{figure}[htbp]
		\begin{center}
			\includesvg[width=0.8\textwidth]{Eksempel1_18-2.svg}
		\end{center}
		\caption{Her kan me sjå tre ulike eksempel av korleis Cech-komplekset ser ut med varierande radius.}
	\end{figure}
\end{example}

\begin{theorem} \label{thm:CASK}
	For $P$ ei endeleg mengda punkt i $\Rb^m$ og $r\in[0,\infty)$, så er Cech komplekset til $P$ med radius $r$ eit abstrakt simplisielt kompleks over $P$.
\end{theorem}

\begin{proof}
	For å visa dette, så må me visa at begge aksioma frå \autoref{def:ASK} held:
	\begin{enumerate}
		\item{ For ein \( \hat{p} \in P \) så ser me at for \( \sigma = \set{\hat{p}} \), så er \( \intersect_{p\in\sigma}\bar{B}_r(p)=\bar{B}_r(\hat{p})\neq\emptyset \) og då er \( \set{\hat{p}} \in \Cech_r(P) \) }
		\item{ Per definisjon av Cech-komplekset, så er alle \( \sigma \in \Cech_r(P) \) ei delmengda av \( P \) }
		\item{ For \( \sigma \in \Cech_r(P) \), så ser me at \( \intersect_{p\in\sigma} \bar{B}_r(p) \neq \emptyset \), men det betyr at for alle \( \tau \subseteq \sigma \), så er \( \intersect_{p\in\tau} \bar{B}_r(p) \) også ikkje-tom, fordi om den var tom, så ville: 
			\[ 
				\intersect_{p\in\sigma} \bar{B}_r(p) = \left( \intersect_{p\in(\sigma\setminus\tau)} \bar{B}_r(p) \right) \intersect \left( \intersect_{p\in\tau} \bar{B}_r(p) \right) = \left( \intersect_{p\in(\sigma\setminus\tau)} \bar{B}_r(p) \right) \intersect \emptyset = \emptyset 
			\] 
			som den ikkje er. Derfor må \( \tau \in \Cech_r(P) \) }
	\end{enumerate}
\end{proof}

Eit fint resultat som er ekvivalent til definisjonen av Cech komplekset:

\begin{theorem}
	For $P$ ei endeleg mengda av punkt i $\Rb^m$ og $r\in[0, \infty)$, så:
	\begin{equation*}
		\sigma\in \Cech_r(P) \Longleftrightarrow \exists x\in\Rb^m; \sigma \subseteq \bar{B}_r(x)
	\end{equation*}
\end{theorem}

\begin{proof}
	($\implies$)
	\\Sidan $A:=\intersect_{p\in\sigma}\bar{B}_r(p)\neq\emptyset$, så er det ein $x\in A$, der for $\forall p\in\sigma: d(p,x)\leq r$, sidan $x\in\bar{B}_r(p)$.
	Men sidan $d(p,x)=d(x,p)$ per definisjon av metrikk. 
	Så $\implies \forall p\in\sigma: p \in \bar{B}_r(x)$
	\\( \( \Longleftarrow \) )
	\\Ved eit likt argument så ser me at  $\forall p \in \sigma : d(x, p) \leq r \implies \forall p \in \sigma : d(p, x) \leq r \implies x \in \intersect_{p \in \sigma} \bar{B}_r(p) \implies \intersect_{p \in \sigma} \bar{B}_r(p) \neq \emptyset$
\end{proof}

Dette gir oss nok grunnlag til å endeleg forstå og definera nerva:

\begin{definition} % Kan definera for vilk. mengda? Og endeleg?
	La $X$ vera ei mengda. Vidare la $F$ vera ei mengda av delmengder av $X$. \emph{Nerva} til $F$, betegna som $\Nc(F)$, er:
	\begin{equation*}
		\Nc(F) := \left \{ \sigma \subseteq F \mid \intersect_{ F_i \in \sigma } F_i \neq \emptyset \right \}
	\end{equation*}
\end{definition}

\begin{theorem}
	For $X$ eit topologisk rom med $F$ ei mengda av delmengder av $X$, då er nerva til $F$ eit abstrakt simplisielt kompleks over $F$.
\end{theorem}

\begin{proof}
	Likt som i beviset for at Cech-komplekset var eit abstrakt simplisielt kompleks (\autoref{thm:CASK}) så må me visa alle betingelsane i \autoref{def:ASK}:
	\begin{enumerate}
		\item{ 
			For \( v \in F \) så har me at for \( \sigma = \set{ v } \) så er \( \intersect_{ p \in \sigma } p = v \neq \emptyset \), så \( \set{v} \in \Nc(F) \) 
		}
		\item{ 
			Alle element av \( \Nc(F) \) er per definisjon ei delmengda av \( F \)
		}
		\item{  
			For \( \sigma \in \Nc(F) \), så ser me at \( \intersect_{v\in\sigma} v \neq \emptyset \), men det betyr at for alle \( \tau \subseteq \sigma \), så er \( \intersect_{v\in\tau} v \) også ikkje-tom, fordi om den var tom, så ville: 
			\[ 
				\intersect_{v\in\sigma} v = \left( \intersect_{v\in(\sigma\setminus\tau)} v \right) \intersect \left( \intersect_{v\in\tau} v \right) = \left( \intersect_{v\in(\sigma\setminus\tau)} v \right) \intersect \emptyset = \emptyset 
			\] 
			som den ikkje er. Derfor må \( \tau \in \Nc(F) \)
		}

	\end{enumerate}
\end{proof}

\begin{example}
	\phantom{abc}
	\begin{figure}[htbp]
		\begin{center}
			\includesvg[width=0.7\textwidth]{Eksempel1_23.svg}
		\end{center}
		\caption{Her har me tre eksempel på mengdar av delmengder i \( \Rb^2 \) og den geometriske realiseringa av nerva deira.}
	\end{figure}
\end{example}

\begin{remark}
	Rett frå definisjonen så ser me at Cech komplekset til $P$ med radius $r$ er ``ekvivalent'' til nerva til $\union_{p \in P} \left \{ \bar{B}_r(p) \right \}$
\end{remark}

Då kan me endeleg utrykka nerveteoremet:

%\begin{definition}
%	La \( Y \) vera eit topologisk rom, og \( X \subseteq Y \) ei delmengda. Og la $A$ vera ei indekseringsmengda. Då er eit \emph{overdekke av $X$ relativ til $Y$} ei samling av delmengdar av $Y$: \( \{ F_\alpha \}_{\alpha\in A} \) der \( F_\alpha \subseteq Y  \) og \( X \subseteq \union_{\alpha \in A} F_\alpha \)
%\end{definition}

%\begin{theorem}
%	(Nerveteoremet for konveks og lukka overdekke) La $V$ vera eit vektorrom med ein topologi og \( F = \set{F_i}_{i=1}^n \) ei endeleg mengda av lukka og konvekse delmengder av $V$.
%	
%	Då er $\gr{\Nc(X)}$ homotopiekvivalent til \( \cup_{i=1}^n F_i \)
%\end{theorem}

%\begin{definition}
%	La \( X \) vera eit topologisk rom, og la \( F \) vera ei mengda av delmengder av \( X \). Då har \( F \) \emph{samantrekkbart snitt} om \( \forall \. \sigma \subseteq F \) så er \( \intersect_{F_i \in \sigma}F_i \) samantrekkbar.
%\end{definition}
%
%\begin{theorem}
%	(Opent, samantrekkbart snitt) La $V$ vera eit topologisk rom (ikkje nødvendigvis eit vektorrom), med \( X \subseteq V \) ei delmengda med underromstopologien. Vidare, la $F$ vera eit endeleg ope overdekke av $X$ relativ til \( X \) med samantrekkbart snitt.
%
%	Då er \( \gr{\Nc(F)} \) homotopiekvivalent til \( X \).
%\end{theorem}

\section{Bevis av Nerveteoremet}

Dette beviset er frå TODO

\begin{definition}
	Gitt eit endeleg abstrakt simplisielt kompleks \( K \) over \( V \), så er den \emph{barysentriske oppdelinga} av \( K \), betegna \( \Sd(K) \), nerva til \( K \).
\end{definition}

\begin{remark}
	Grunnen til at me betegnar den barysentriske oppdelinga til \( K \) for \( \Sd(K) \) og ikkje \( \Nc(K) \) er fordi det er praktisk å skilja dei, og enklare å lesa. I tillegg så blir den barysentriske oppdelinga av eit abstrakt simplisielt kompleks ofta kalla for ein "<subdivision>" på engelsk, og derfor brukar me \( \Sd \).
\end{remark}

\begin{lemma}
	For \( K \) eit endeleg abstrakt simplisielt kompleks over \( V \), så har me at:
	\( \gr{\Sd(K)} \) er homeomorf til \( \gr{K} \).
\end{lemma}

\begin{proof}
	TODO
\end{proof}

\begin{example}
	Teikn
\end{example}

\begin{definition} \label{thm:Gamma} % Kan fjerna phi og f avhengigheita frå notasjonen. U, phi, f avhengigheita er implisitt.
	Gitt \( U = \set{U_i}_{i=1}^n \) ei endeleg mengda av konvekse mengder i \( \Rb^m \). Indekser elementa i \( \Sd(\Nc(U)) \) etter \( \sigma_1, \sigma_2, \dots, \sigma_k \). For \( x \in \gr{\Sd(\Nc(U))}_\phi \)  med dei unike ref:TODO utvida barysentriske koordinatane \( a_1, a_2, \dots, a_k \).
	
	Vidare, for kvar \( \sigma \in \Nc(U) \) velg punkt \( v_\sigma \in \intersect_{u \in \sigma} u \). 

	La då:
	\[
		\Gamma(x) = \sum_{i=1}^k a_iv_{\sigma_i}
	\]
\end{definition}

\begin{lemma} % Må visa veldefinert mhp f og phi?
	\( \Gamma \) frå \ref{thm:Gamma} er kontinuerleg og biletet er i \( \union_{u\in U} u \).
\end{lemma}

\begin{proof}
	TODO
\end{proof}

\begin{example}
	Teikn
\end{example}

\begin{definition} % Definér open ball?
	For ei mengda \( U_i \) og ein \( \epsilon > 0 \) definér \( U_i^\epsilon := \union_{x \in U_i} B_\epsilon(x) \), og for ei endeleg mengda av mengder \( U=\set{U_i}_{i=1}^n \) definér \( U^\epsilon := \set{U_i^\epsilon}_{i=1}^n \).
\end{definition}

\begin{lemma} \label{thm:epsilondekke} % Epsilondekke
	For alle endelege mengder av kompakte, konvekse delmengder av \( \Rb^d \), \( U = \set{U_i}_{i=1}^n \), så \( \exists \epsilon > 0 \) sånn at \(f: \Nc(U) \to \Nc(U^\epsilon) \) som tek 
	\[ 
		\set{U_{i_1}, U_{i_2}, \dots, U_{i_k}} \mapsto \set{U_{i_1}^\epsilon, U_{i_2}^\epsilon, \dots, U_{i_k}^\epsilon} 
	\] 
	er ein bijeksjon.
\end{lemma}

\begin{proof}
	TODO
\end{proof}

\begin{definition} \label{thm:psi} % Kva er betingelsane på overdekket her?
	For ei endeleg mengda av lukka, kompakte delmengder \( U_i \subseteq \Rb^d \), med \( U = \set{U_i}_{i=1}^n \), velg ein \( \epsilon \) som i \ref{thm:epsilondekke}, kan då danna:
	\[
		\phi_i(x) := \frac{d(x, \Rb^m \setminus U_i^\epsilon)}{d(x, U_i) + d(x, \Rb^m \setminus U_i^\epsilon)}
	\]
	Og vidare:
	\[
		\psi_i(x) := \frac{\phi_i(x)}{\sum_{k=1}^n \phi_k(x)}
	\]
\end{definition}

\begin{lemma} % Burde bruka alle
	For ei endeleg mengda av lukka, kompakte delmengder \( U_i \subseteq \Rb^d \), med \( U = \set{U_i}_{i=1}^n \) , la \( \psi_i \) og \( \epsilon \) vera definert som i \ref{thm:psi}. Då er følgande sant:
	\begin{enumerate}
		\item{For \( x \in  U_i \), då er \( \psi_i(x) \geq \psi_j(x) \forall j \) }
		\item{For \( x \in \Rb^d \setminus U_i^\epsilon \), då er \( \psi_i(x)=0 \) }
		\item{For \( x \in \union_{u \in U} u \), så er \( \sum_{i=1}^n \psi_i(x) = 1 \) }
		\item{For \( x \in \union_{u \in U} u \), så er \( \psi_i(x) \geq 0 \) }
	\end{enumerate}
\end{lemma}

\begin{proof}
	TODO
\end{proof}

\begin{definition} \label{thm:Psi}
	For ei endeleg mengda av lukka, kompakte delmengder \( U_i \subseteq \Rb^d \), med \( U = \set{U_i}_{i=1}^n \). Vidare la \( \psi_i \) vera som i \ref{thm:psi}. Definér for \( x \in \union_{u \in U} u \):
	\[
		\Psi(x) := \sum_{i=1}^n \psi_i(x)f(U_i)
	\]
\end{definition}

\begin{lemma} % Noko meir?
	\( \Psi \) frå \ref{thm:Psi} er kontinuerlig, og har bilete i \( \gr{\Nc(U)}_f \).
\end{lemma}

\begin{definition}
	bst
\end{definition}

\begin{lemma}
	im bst-i for gamma inne i U-i
\end{lemma}

\begin{lemma}
	im U-i inne i bst-i for psi
\end{lemma}

\begin{lemma} % Sløyfast?
	psi gamma homotopiekvivalent
\end{lemma}

\begin{theorem}
	Tek spesiell overdekke
	Gamma psi er homotopiekvivalen
	psi gamma

\end{theorem}

\section{Anvendingar av Nerveteoremet}

\end{document}
